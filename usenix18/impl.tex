\vspace{-0.1in}
\section{Implementation}
\label{sec:impl}
We discuss some  implementation details of \tool.
The  \tool compiler is written in Python and
compiles each of our benchmarks in under a second to C++ code that
makes calls to the ABY
library~\cite{aby}. 
ABY provides a GMW-based~\cite{gmw} and a  Yao-based~\cite{yao} protocol for boolean computations.
Although \tool can generate code for both, we have observed better performance when using the latter
and use it in our evaluation.
We use the default ABY configuration for our
experiment\footnote{We set the security parameter to 128 and use OT
extension-based arithmetic multiplication triples generation.}. ABY
provides multi-threading support (for the offline phase of the \mpc
protocol); we leverage the support and use at most four threads in our
evaluation. 

\tool programs can have the following operators:
addition, subtraction, multiplication, division by powers of two, left
shift,
logical and arithmetic right shifts, bitwise-(and, or, xor), unary
negation, 
bitwise-negation, logical-(not, and, or, xor), and comparisons (less
than, greater than, equality).  
Because of their high cost, integral division and floating-point
operators are not supported natively by \tool. 
However, we have implemented integral division in 30 lines of \tool,
while the floating-point support in ABY is under active
development~\cite{ddkssz15}.

Some of our benchmarks require accessing arrays at secret
indices. While \tool enforces the array indices to be
public, secret indices can be encoded in \tool using multiplexers.
For example, consider the expression $A[x]$ where $A$ is an array of
size 2 and $x$ is secret-shared. The developer can express
this functionality in \tool as $\cond{x > 0}{A[1]}{A[0]}$. In general,
a secret access to an array of
size $n$ requires ${n}-1$ multiplexers in \tool.

%Moved this to Section 4 since P_i and Q_i are ill defined here
%Secure code partitioning currently requires a manual step.
%In particular, the developer needs to decompose a program $P$ into
%$P_1||\ldots||P_k$ manually (Section~\ref{sec:pipe}). Then \tool
%generates $Q_1||\ldots||Q_k$ (the sub-programs with appropriate
%wrappers) automatically.
%Automating the decomposition step requires an analysis that can 
%statically estimate the resource usage of an \tool program. We leave
%the construction of such an analysis as future work.

We use an off-the-shelf solver
(SeaHorn~\cite{seahorn}) to bound check the array indices
($\models e < n$
in ({\sc {T-Read}}) and ({\sc{T-Write}}),
\figureref{compile}). We take the \tool source program and
translate it as an input to the solver. The solver takes less
than a minute on our largest benchmark to verify that all the array
accesses are in-bounds.

Our implementation assigns the type labels (rule~{\sc{T-Decl}})
conservatively. Only the variables that govern the control flow, i.e.,
variables in \kw{if}-conditions and \kw{for}-loop counters are
assigned public labels.
All other variables are assigned arithmetic labels (that can later be
coerced to boolean).
We leave a more sophisticated type inference procedure for future work.

The compilation rules of \figureref{compile} can introduce
repeated coercions from arithmetic to
boolean and vice versa.
Since \tool is aware of the cryptographic costs associated with these coercions,
it tries to minimize them using several optimizations.
For instance, if boolean shares of a variable are available in scope and the variable is involved in
an addition then \tool performs the addition in boolean rather than first coercing boolean to arithmetic
and then performing an arithmetic addition. However, multiplication is always performed in arithmetic
as the cost of a boolean multiplication is much higher than the cost of coercing from boolean to arithmetic, performing an arithmetic multiplication, and then coercing from arithmetic to boolean~\cite{aby}.
Other optimizations include the standard ``common subexpression elimination"
optimization~\cite{dragonbook}.
On each coercion, \tool memorizes the pair of arithmetic
share and boolean share involved in the coercion. 
\tool invalidates such pairs when the variables corresponding to the
shares are overwritten by assignments. 
In subsequent coercions, \tool  reuses valid pairs (if available)
instead of inserting code to recompute them afresh.

Since all these optimizations are standard compiler
optimizations~\cite{dragonbook}, we rely on their correctness to
maintain the correctness and security for the optimized programs.

