\section{Conclusion and Future Work}
\label{sec:conclude}

We presented \tool, the first
cryptographic cost aware framework that generates efficient and
scalable \mpc protocols from high-level programs.
The generated protocols comprise combinations
of arithmetic and boolean circuits and have performance comparable to, or better than the previously known custom specialized
protocols from previous works.
The compiler is backed by formal semantics that help 
%it
%insert type coercions between boolean and arithmetic parts automatically while 
it maintain correctness, security,
and efficiency.

Currently, we are working on a front-end to translate Tensorflow code
to \tool. The aim here is to provide a push button implementation
that generates secure implementations for existing Tensorflow models. 
In the future, we would like to extend our security guarantees to
malicious adversaries. We think that, with suitable
changes to operator costs, \tool can target other backends including 3PC or
MPC, or protocols secure against malicious adversaries.
 The cryptographic backends continue to improve and the modular design of \tool makes it easy to integrate with the best available backends.
However, we are currently unaware of a maliciously secure \mpc implementation for combinations of
arithmetic and boolean circuits.
Finally, we will explore the possibility of mechanically verifying the  compiler implementation.






%% \section{Conclusion}
%% \label{sec:conclude}
%% We presented \tool, the first
%% cryptographic-cost aware framework that generates efficient and
%% scalable \mpc protocols from high-level programs.
%% The generated protocols comprise combinations
%% of arithmetic and boolean circuits and have performance comparable to, or better than the custom, specialized
%% protocols from previous works.
%% The \mpc backend we use provides security against semi-honest or passive adversaries.
%% %our compiler is oblivious to the ``type of adversary'' for the underlying \mpc backend.
%% Given a \mpc backend secure against malicious or active adversaries, Theorem \ref{theorem:security} 
%% can be extended to obtain a framework with malicious security. Furthermore,  
%% partitioning can be modified to provide malicious security by {\it signing} the shares of the intermediate states. However, we are not aware of efficient maliciously secure \mpc implementation for combinations of
%% arithmetic and boolean circuits and hence leave a complete treatment of malicious security to future work.
%% % Several areas of improvement remain: a) While the backend that we use~\cite{aby} only provides security against a semi-honest adversary, our compiler is oblivious to the ``type of adversary'' for the underlying backend cryptographic protocol. In particular, if our cryptographic backend were maliciously secure, then Theorem \ref{theorem:security} would provide us a framework with malicious security. Furthermore, modifying our code partitioning to provide malicious security is fairly easy (one would instead provide {\em signed} shares of intermediate states in the decomposed programs to obtain malicious security). b) Our compiler converts programs into circuits before executing a \mpc and is efficient for the applications we present. However, one could consider executing programs that exploit secret-dependent memory accesses (such as binary search) to obtain better performance (as in \cite{oblivm}). It would be interesting to see if our techniques can be combined with that of \cite{oblivm}. \nc{Too much? Should we simply say only about the malicious security and stop?}
