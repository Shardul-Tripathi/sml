\vspace{-0.1in}
\section{Implementation}
\label{sec:impl}
We discuss some  implementation details of \tool.
The  \tool compiler is written in Python and
compiles each of our benchmarks in under a second to C++ code that
makes calls to the ABY
library~\cite{aby}. 
ABY provides support for Arithmetic computations based on \cite{autoS}, and boolean computations based on GMW~\cite{gmw} as well as Yao's garbled circuits~\cite{yao}. Although \tool can generate code for both kinds of boolean computations, we have observed better performance when using  garbled circuits
and use it in our evaluation. Hence, \tool generated code uses Arithmetic computations and garbled circuits based boolean computations.
We use $128$ bits of security and OT
extension-based arithmetic multiplication triplets generation. ABY
provides multi-threading support (for the offline phase of the \mpc
protocol); we leverage the support and use at most four threads in our
evaluation. 

\tool programs can have the following operators:
addition, subtraction, multiplication, division by powers of two, left
shift,
logical and arithmetic right shifts, bitwise-(and, or, xor), unary
negation, 
bitwise-negation, logical-(not, and, or, xor), and comparisons (less
than, greater than, equality).  
Because of their high cost, integral division and floating-point
operators are not supported natively by \tool. 
However, we have implemented integral division in 30 lines of \tool,
while the floating-point support in ABY is under active
development~\cite{ddkssz15}.

While most ML applications make data oblivious passes over the data,
some of our benchmarks do require accessing arrays at secret
indices. \tool enforces the array indices to be
public, but secret indices can be encoded using multiplexers.
For example, consider the expression $A[x]$ where $A$ is an array of
size 2 and $x$ is secret-shared. The developer can express
this functionality in \tool as $\cond{x > 0}{A[1]}{A[0]}$. In general,
a secret access to an array of size $n$ requires ${n}-1$ multiplexers
in \tool.

%Moved this to Section 4 since P_i and Q_i are ill defined here
%Secure code partitioning currently requires a manual step.
%In particular, the developer needs to decompose a program $P$ into
%$P_1||\ldots||P_k$ manually (Section~\ref{sec:pipe}). Then \tool
%generates $Q_1||\ldots||Q_k$ (the sub-programs with appropriate
%wrappers) automatically.
%Automating the decomposition step requires an analysis that can 
%statically estimate the resource usage of an \tool program. We leave
%the construction of such an analysis as future work.

We use an off-the-shelf solver
(SeaHorn~\cite{seahorn}) to check that the array indices
are within bounds ($\models e < n$
in ({\sc {T-Read}}) and ({\sc{T-Write}}),
\figureref{compile}). We take the \tool source program and
translate it as an input C program to the solver. The solver takes less
than a minute on our largest benchmark to verify that all the array
accesses are in-bounds. This C program also enables
validation of \tool generated protocols via differential testing~\cite{mckeeman,frigate}.

Our implementation assigns the type labels (rule~{\sc{T-Decl}})
conservatively. Only the variables that govern the control flow, i.e.,
variables in \kw{if}-conditions and \kw{for}-loop counters are
assigned public labels.
All other variables are assigned arithmetic labels (that can later be
coerced to boolean). While assigning public labels to control
flow variables is critical for
performance and/or security, assigning arithmetic labels to other
variables incurs minimal cost in practice.
We leave a more sophisticated type inference procedure for future work.

The compilation rules of \figureref{compile} can introduce
repeated coercions from arithmetic to
boolean and vice versa.
Since \tool is aware of the cryptographic costs associated with these coercions,
it tries to minimize them using several optimizations, e.g., by
%For instance, if boolean shares of a variable are available in scope and the %variable is involved in
%an addition then \tool performs the addition in boolean rather than first %coercing boolean to arithmetic
%and then performing an arithmetic addition. 
%However, multiplication is always performed in arithmetic as the cost of a %boolean multiplication is much higher than the cost of coercing from boolean %to arithmetic, performing an arithmetic multiplication, and then coercing from %arithmetic to boolean~\cite{aby}.
%Other optimizations include
the standard ``common subexpression elimination"
optimization~\cite{dragonbook}.
On each coercion, \tool memorizes the pair of arithmetic
share and boolean share involved in the coercion. 
\tool invalidates such pairs when the variables corresponding to the
shares are overwritten by assignments. 
In subsequent coercions, \tool  reuses valid pairs (if available)
instead of inserting code to recompute them afresh.
These optimizations are standard compiler
optimizations~\cite{dragonbook}, and we rely on their correctness
(optimizations preserve outputs and well-typedness) to
maintain the security of the optimized programs.

\noindent\textbf{Implementing automatic code partitioning.}
\tool decomposes the source program
$\prog$ into a sequence of small programs
$\prog_1\seq\dots\seq\prog_k$, and then appropriately instruments them
to produce $\progn_1,\dots\progn_k$, as detailed in
Section~\ref{sec:pipe}. It then
compiles and executes the $k$ programs
$\progn_1,\progn_2,\ldots,\progn_k$ sequentially, freeing up memory
usage after execution of each $\progn_i$.
Automating the  decomposition step requires an analysis that can 
statically estimate the resource usage of a \tool program.
Resource analysis of high-level programs is a well-known hard problem~\cite{raml}
and we describe a heuristic analysis.

To build $s_1$, we consider the longest
prefix of $s$ whose computation {\it size} is below the threshold enforced by the available memory of the machine.
If $\prog = \prog_1; \prog_r$ then we recurse on $\prog_r$ to obtain $\prog_2,\ldots,\prog_k$.
For a program $\mprog$, to estimate ${\it size}(\mprog)$, we need to discuss three important cases: if $\mprog\equiv\mprog_1;\mprog_2$
then ${\it size}(\mprog)={\it size}(\mprog_1)+{\it size}(\mprog_2)$; if $\mprog\equiv\ite{e_1}{\mprog_1}{\mprog_2}$ then ${\it size}(\mprog)=\max({\it size}(\mprog_1),{\it size}(\mprog_2))$; if $s\equiv\forl{i}{n_1}{n_2}{\mprog_1}$ then
${\it size}(\mprog)=(n_2-n_1){\it size}(\mprog_1)$. If $(n_2-n_1){\it size}(\mprog_1)$ is above the threshold then
we replace $\mprog$ by 
\[\forl{i}{n_1}{\frac{n_1+n_2}{2}}{\mprog_1}; \forl{i}{\frac{n_1+n_2}{2}}{n_2}{\mprog_1}\]
 and recurse to find the prefix again.
This heuristic analysis is sufficient for the benchmarks discussed in
our evaluation.
