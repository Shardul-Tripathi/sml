%\newpage

\section{Description of benchmarks}\label{app:benchmarks}

\subsection{Benchmarks}

We use $[N]$ to denote $\{0,1,\dotsc, N-1\}$. Further, given a vector
$x\in\R^d$, we say $\mathsf{argmax}\ x = i$ if $x_i =
\mathsf{max}\ \{x_0,\ldots,x_{d-1}\}$. Finally, if $A$ is
a matrix (resp. vector) then we write $f(A)$ for the matrix (resp. vector)
obtained by applying the scalar function $f$ to each entry of $A$
pointwise.

We focus on the machine
learning models for {\it classification}. A classifier $C$ uses a
trained model to {\it predict} a
label $\ell$ for an input data point $x$. For example, given a
data point which is a tuple of humidity and temperature 
a classifier can predict a label ``will rain'' or ``will not rain''.
The {\it model size} of a classifier is the number of parameters in the model.
For example, the model size of the classifier in~\figureref{ex-sml} is $|w|+1=31$.
The {\it
  accuracy} of a classifier refers to the fraction of data points that
the classifier labels correctly from a given set of test data
points.

\paragraph{Standard classifiers.}
A {\em binary linear classifier} is one of the simplest classifiers. Here,
the input is a data point $x\in\R^d$,
and the model is a vector $w\in\R^d$.
 The possible labels are
$\ell\in \{ true, false \}$ and the classifier is
$C_w\equiv w^Tx>0$.
%, where $w^T$ denotes the transpose
%of matrix $w$, $\matmul$ denotes matrix multiplication (with a vector in this case) and $a>b$ returns %$\mathit{true}$ if $a>b$ and $\mathit{false}$ otherwise. 
%The {\it model size} of this classifier, i.e., the number of parameters in the model is $|w|$, i.e., $d$.
%This classifier requires $d$ multiplications, $d-1$
%additions, and a single comparison.
%
A more interesting classifier is {\em Na\"{i}ve Bayes}~\cite{shafindss} that predicts labels
from the set $[n]$.
Here, the input data point is a {\it feature}
vector $x=(x_0,x_2,\ldots,x_{d-1})^T$ where each $x_j\in [F]$.
%The model has two matrices: a vector $P$ of length $n$ s.t. for each
%$i
%\in [n]$, $P(i)$ is the log-probability that the output label is $i$.
%$P(i) \equiv \log p(\ell=i)$, where $p(\ell=i)$ is the likelihood
%that the output label is $i$ for $i\in[n]$.
%The other matrix $T$ has size $d\times F\times n$ and for each $j\in
%[d], k\in [F], i\in [n]$, the entry $T(j)(k)(i)$ is 
%\equiv\log p(x_j|\ell=i)$, i.e.,
%the log-probability that  the $j^{th}$ input feature is $k$  conditioned on  the output label being $i$.
%The classifier $C_{P,T}(x)$ outputs
%\[
%{\sf argmax}\ P+\sum_{j=0}^{d-1} T(j)(x_j)
%\]
%The additions in the expression above sum $n$-dimensional vectors.
The model size of this classifier is $\Theta(ndF)$.
% comparisons and additions. %Note here  that computing $T(j)(x_j)(i)$ requires a secret look-up in matrix  $T$  based on secret input $x_j$ resulting in $F-1$ comparisons using multiplexers (see
  %Section~\ref{sec:impl}). %\nc{Made DG's comment as part of regular text - worth point out}. \cmmt{Is there a +1 etc missing or is this exact?} \nc{By our calculation in Sec 5, it seems that this should be $F/2$ multiplexers}
%
A {\em decision tree} of size $N$ and depth $d$ takes as input an $x\in\R^d$
%,
 and 
%the
%model consists of a binary tree of depth $d$ with a boolean predicate
%assigned to each internal node.
% The root note is the $0^{th}$ node and
%for an internal node $j \in [N]$, the children nodes are $2j+1$ and
%$2j+2$. Each internal node $j\in[N]$ at depth $i$ has a predicate
%$b_{j}^i\equiv x_i\leq w_{j}$. We start evaluating the tree from the
%root and if the predicate at the current node $b_j^i$ is false (resp.,
%true) then $x$ is passed to the left (resp., right) child with
%predicate $b_{2j+1}^{i+1}$ (resp., $b_{2j+2}^{i+1}$). This process is
%repeated till we reach a leaf. The leaves of the tree are labels and
%the output label is the leaf visited by this traversal.
%Such a (binary) decision tree 
%and can be encoded as a polynomial (of
%degree linear in depth of the tree) and then
 the prediction task
reduces to evaluation of a $d$-degree polynomial~\cite{shafindss}.
% E.g., a binary tree with depth one
%and size $N=3$ can be encoded as the polynomial
%$(1-b(x))\ell_0 + b(x)\ell_1$ where $b(x)$ is the predicate at the
%root, $\ell_0$ is the label of the left leaf and $\ell_1$ is the label
%of the right leaf.

\paragraph{Deep neural nets.}
The next class of classifiers that we benchmark are deep neural nets
or DNNs. A DNN has multiple layers such that each layer computes a
matrix
multiplication followed by an {\it activation} function $f$. The most
common activation functions are square $f(x)=x^2$ and rectifier linear
unit (ReLU) $f(x)=\mathsf{max}(x,0)$.
Given an input vector $x$, the predicted label of a DNN is
\[
 \mathsf{argmax}\ W_N\matmul f_{N-1}(\ldots f_1(W_1\matmul x)\ldots)
\]
Here, $f_i$'s are the (public) activation functions, the model
consists of matrices $W_i$,  $x \in \R^d$ is the input vector, and the
operator $\matmul$ denotes a matrix multiplication.
Neural nets usually have one or more fully connected layers, each of
which multiplies a matrix with a vector.
Some neural nets have convolution layers and such DNNs are also called
Convolutional Neural Nets or CNNs.
For the purpose of this paper, a convolution can be considered as a
(heavy) matrix-matrix multiplication. The size of matrices manipulated
by a convolution layer grows linearly with {\it window size}
(typically 9 or 25), the number of {\it output channels} (typically
16, 32, or 64), and the size of the matrix input to this layer.
Therefore, fully connected layers are lighter computation-wise compared
to convolution layers. However, the model size  of fully connected
layers is larger than those of convolution layers.
In general, DNNs are computationally heavy but provide
much better accuracies on computer vision tasks than the classifiers
discussed above.

\paragraph{State-of-the-art classifiers.}
Finally, there are a class of machine learning classifiers that are
much more efficient than
DNNs and provide reasonably good accuracies on standard learning
tasks. \bonsai~\cite{bonsai} is a state-of-the art classifier in this
class and \tool provides the first 2PC protocol for it.
\bonsai takes as input $x \in \R^d$, and its model consists of a
binary tree with $N$ nodes, and a matrix $Z$. Each node $j$ contains
matrices $W_j$ and $V_j$, and a vector $\theta_{j}$. The internal node
$j$  evaluates a predicate $(\theta_j^T \matmul Z \matmul x) > 0$ to decide whether
to pass $x$ to the left child $2j+1$ or the right child $2j+2$.
The predicted value is

%\aseem{I had a hard time parsing this. (1) Should we use $\cdot$ for
%  matrix multiplication as we did for DNNs? (2) Precedence of $\circ$
%  over scalar multiplication (or is it associative so doesn't matter?)
%(3) The argument to $f$ is a scalar (right?), and so is its output,
%so then is $\circ$ defined OR is it applying $f$ pointwise to a vector?}
%\nc{(1) I agree we should do that to be consistent with before. (2) I assumed $f()$ is computed to give a vector somehow and $I_j(x)W_j^TZx$ also gives a vector somehow and then we do $\circ$; I might be wrong. (3) Hmm.. I thought it was a vector, but I dont know how. (4) Can we specify somewhere the dimensions of these matrices?}
%\vspace{-10pt}
\[
{\sf argmax} \sum_{j = 0}^{N - 1} I_j(x)[ (W_j^T\matmul Z\matmul x) \circ (f(V_j^T\matmul Z\matmul x))] 
\]
Here, $I_j(x)$ is 1 if the $j^{th}$ node is present on the path traversed by $x$
and is zero otherwise. 
The operation $\circ$ is a pointwise multiplication of two vectors, $W_j$'s and $V_j$'s
are matrices of appropriate dimensions. The activation function $f$ is given by $f(y) = y$ if
$-1 < y < 1$ and $\mathrm{sign}(y)$ otherwise.
%% ($f$ on vectors and
%% matrices is defined point-wise on every element of the
%% vector/matrix).
%\bonsai can be
%seen as a variant of decision trees where the prediction is a function
%of the path traversed from root to the leaf and not just the leaf
%itself.

We implement these classifiers in \tool and report
the time taken for making secure predictions. Ideally, the machine
learning classifiers are mathematical expressions over $\R$ that are
usually approximated by floating-point operations. 
As is standard, we port the classifiers to integer manipulating
programs by scaling the models and rounding~\cite{minionn}. These
ported classifiers are then implemented in \tool.