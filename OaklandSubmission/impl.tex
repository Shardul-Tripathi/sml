\section{Implementation}
\label{sec:impl}
The  \tool compiler is written in Python and
compiles each of our benchmarks in under a second to C++ code that makes calls to the ABY
library~\cite{aby}. We set the security paramenter to 128 and use OT extension-based arithmetic multiplication triples generation. ABY provides facilities to use multiple threads\footnote{The protocol has an offline phase and an online phase. Although multiple threads can be used in the offline phase, the online phase is single threaded.}. We use at most four threads in our evaluation.

The current language implementation supports the following operators:
addition, subtraction, multiplication, division by powers of two, left shift,
logical and arithmetic right shifts, bitwise-(and, or, xor), unary negation,
bitwise negation. Because of its high cost, general division is not supported natively by \tool.
However, it can be implemented in about 30 lines.

We use an off-the-shelf solver
(SeaHorn~\cite{seahorn}) to check the verification conditions required
to prove that the array indices remain in bound ($\Gamma\vDash e < n$ in {\sc T-Read}, Figure~\ref{fig:compile}). The solver takes less
than a minute on our largest benchmark.
The type label is assigned conservatively in  {\sc T-Decl}.
Only the variables that govern the control flow, i.e., variables
in if-conditions and for-loops are assigned public labels.
All other variables are assigned arithmetic labels.
In the future, we would like to explore more intelligent strategies for label assignment.

Some of our benchamrks require accessing arrays at secret location.
For example, consider the expression $A[x]$ where $A$ is an array of size 2 and $x$ is shared.
\tool enforces the array indices to be public. The developer expresses this functionality in \tool
as $x > 0 ? A[1] : A[0]$. In general, a secret access to an array of size $2^n$ requires $2^{n}-1$ multiplexers in \tool. Secure code partitioning currently requires a manual step.
In particular, the developer needs to decompose a program $P$ into $P_1||\ldots||P_k$ manually (Section~\ref{sec:pipe}). Then \tool generates $Q_1||\ldots||Q_k$ automatically.
Automating the decomposition step requires an analysis that can 
statically estimate the resource usage of a \tool program and we leave the construction of such
an analysis as future work.

The compilation rules of Figure~\ref{fig:compile} can introduce redundant coercions from arithmetic to
bool and vice versa.
The \tool compiler performs several optimizations to reduce these.
For example, on each coercion, \tool memorizes the pair of arithmetic share and boolean share involved in the coercion.
\tool invalidates such pairs when the variables corresponding to the shares are overwritten by assignments.
In subsequent coercions, \tool can reuse valid pairs (if available) instead of recomputing them afresh. 

\divya{
\begin{enumerate}
\item add all the expressions that our language supports such as mult, bit-wise and and anything else.
\item Say that we compile to source to intermediate and use inter to circuit of ABY.. We can decide if we want to say this.
\end{enumerate}
}
 