
\subsubsection*{Security theorem} The protocols we generate provide
simulation-based security against a semi-honest adversary, in the
framework of ~\cite{gmw,can00,can01}. At a very high level, in this
framework, parties are modeled as non-uniform interactive turing
machines (ITMs), with inputs provided by an environment $\env$. An
adversary $\adv$, selects and ``corrupts'' one of the parties -
however, $\adv$ still follows the protocol specification. $\adv$
interacts with $\env$ that observes the view of the corrupted
party. At the end of the interaction, $\env$ outputs a single bit based on the output of the honest party and the view of the adversary. Two
different interactions are defined: the {\em real world} and an {\em
  ideal world}. In the real interaction, the parties run the protocol
$\prot$ in the presence of $\adv$ and $\env$. Let
$\real_{\prot,\adv,\env}$ denote the  distribution ensemble
describing $\env$'s output in this interaction. 
In the ideal interaction, parties send their inputs to  a
trusted functionality that performs the desired
computation truthfully. Let $\simu$ (the simulator) denote the
adversary in this ideal execution, and $\ideal_{\F,\simu,\env}$
the distribution ensemble describing $\env$'s output after
interacting with the ideal adversary $\simu$. A
protocol $\prot$ is said to {\em securely realize} a functionality
$\F$ if for every adversary $\adv$ in the real interaction, there is
an adversary $\simu$ in the ideal interaction, such that no
environment $\env$, on any input, can tell the real interaction apart
from the ideal interaction, except with negligible probability (in the
security parameter $\secparam$). More precisely, the above two
distribution ensembles  are computationally indistinguishable. 

We shall assume a cryptographic \mpc backend that securely implements
any circuit $\crct$ that is output by our compiler
(see \figureref{circuits}). In more detail, this means that for any
well-typed source program $s$, let $\crct$ be the output circuit output
(as in \theoremref{correctness}). We assume that there
exists a \mpc protocol $\prot$ that securely realizes the
functionality $\crct$ and let $\simu_{\tiny{2pc}}$ be the
corresponding simulator  (that runs on $\crct$, input of the corrupt
party and the output obtained from trusted functionality for
$\crct$). We note that ABY~\cite{aby} provides such a protocol
$\prot$ and simulator $\simu_{\tiny{2pc}}$ for all circuits $\crct$
output in our framework. We now state and prove our
security theorem. 
 
\begin{theorem}[Security]\label{theorem:security}
Let $s$ be any functionality or program in our source language with
outputs (or observations) $O_1$, that outputs a circuit
$\chi$ (as defined in Theorem \ref{theorem:correctness}). Let protocol
$\prot$ be the two-party secure computation protocol that securely
realizes $\chi$ (as defined above). Then, $\prot$ securely realizes
$s$.
\end{theorem}

\noindent {\em Proof.} 
Our simulator $\simu$ simply runs our compiler on program $s$ to obtain $\crct$. It is crucial that this compilation to circuits does not require the secret inputs of the parties. Next, $\simu$ sends the input of the corrupt party to the trusted functionality of $s$ to obtain outputs $O_1$. Note that $O_1$ is same as the observations in the source semantics. 
By \theoremref{correctness}, these outputs $O_1$ are identical to outputs (or observations) $O_2$ of $\crct$ under circuit semantics. 
Next, $\simu$ runs $\simu_{\tiny{2pc}}$ on $\crct$, input of the corrupt party and $O_2$. 
From the security of $\prot$, we have that the simulated view output by $\simu_{\tiny{2pc}}$ is indistinguishable from the real view of the $\prot$ for $\crct$ between the two parties. Hence, the security follows.
 
% and upon receiving output $O_1$, executes $\simu_{\tiny{2pc}}$ on $\chi$ and $O_1$. Let $O_2$ be the outputs (or observations) of $\chi$. First, from Theorem \ref{theorem:correctness}, we have that the observations in $s$ and $\chi$ are the same (i.e. $O_2 = O_1$). Now, from the security of $\prot$, we have that the simulated view output by $\simu_{\tiny{2pc}}$ (with circuit $\chi$ and output $O_2$) is indistinguishable from the real view of the $\prot$ (when executed on circuit $\chi$ and when the output is $O_2$). Combining these two statements, the proof of the theorem follows. 
  %% \\
  %% \[
  %% \begin{array}{rrcl}
  %%   \ftext{Value} & v &::=& c \mid [\overline{c_{i}}]_{n}\\
  %%   \ftext{Observation} & O &::=& \cdot \mid c; O\\
  %% \end{array}
  %% \]
