\newcommand{\kw}[1]{{\lstinline[basicstyle=\small\color{blue}]{#1}}}
\newcommand{\ftext}[1]{\text{\small{#1}}}
\newcommand{\cond}[3]{\ensuremath{{{#1}\:?\:{#2}\::{#3}}}}
\newcommand{\for}[4]{\ensuremath{\kw{for}\:{#1}\:\kw{in}\:[{#2}, {#3}]\:\kw{do}\:{#4}}}
\newcommand{\ite}[3]{\ensuremath{\kw{if}({#1}, {#2}, {#3})}}
\newcommand{\loops}[3]{\ensuremath{\kw{while}\:{#1} \leq {#2}\:\kw{do}\:{#3}}}

\section{Formal development}
\label{sec:ld}

In this section we prove the correctness and security of our compiler.
%
We first formalize the source and target languages. Our source runtime
semantics is a model of the ideal, trusted third-party semantics, and
generates observations corresponding to the values revealed to the
parties.
%
The target language semantics (a model of the C++ code generated by
our compiler implementation) ``computes away'' the public parts of the
compiled program, generating a secure computation circuit.
%
Finally, we formalize the circuit semantics that computes the
generated circuit, and like the source semantics, emits observations.

We then present the compilation judgments. To prove the correctness
of our compiler, we prove that it preserves the observations. For
security of our compiler, we reduce the security argument to the
security of the cryptographic protocol used to compute the secure
computation circuit.

We present only selected parts of our formalization for space
reasons. Full definitions and proofs can be found in the supplementary
material submitted along with the paper.


\begin{figure}
  \small
  \[
  \begin{array}{rrcl}
    \ftext{Base type} & \sigma &::=& \kw{uint} \mid \kw{bool}\\
    \ftext{Type} & \psi &::=& \sigma \mid \sigma[n]\\
    \ftext{Constant} & c &::=& n \mid \top \mid \bot\\
    \ftext{Expression} & e &::=& c \mid x \mid e_{1} + e_{2} \mid e_{1} > e_{2} \mid \cond{e_{1}}{e_{2}}{e_{3}}\\
    & &\mid& [\overline{e_{i}}]_{n} \mid x[e] \mid \kw{in}_{j}\\
    \ftext{Statement} & s &::=& \psi\:x = e \mid x := e \mid \for{x}{n_{1}}{n_{2}}{s}\\
    & & \mid& x[e_{1}] := e_{2} \mid \ite{e}{s_{1}}{s_{2}} \mid \kw{out}\:e \mid s_{1}; s_{2}\\
    & & \mid& \loops{x}{n}{s}
  \end{array}
  \]
\caption{Source language}
\label{fig:srclang}
\end{figure}

\begin{figure}
  \small
  \fbox{$\rho \vdash e \Downarrow v$}
  \[
  \\
  \begin{array}{c}
    \inferrule*[lab={\footnotesize{E-Var}}]
               {
               }
               {
                 \rho \vdash x \Downarrow \rho(x)
               }
               
               \hspace{0.1cm}
               
    \inferrule*[lab={\footnotesize{E-Add}}]
               {
                 \forall i \in \{1, 2\}.\:\rho \vdash e_{i} \Downarrow n_{i}
               }
               {
                 \rho \vdash e_{1} + e_{2} \Downarrow n_{1} + n_{2}
               }

               \hspace{0.1cm}
               
    \inferrule*[lab={\footnotesize{E-Read}}]
               {
                 \rho \vdash x \Downarrow [\overline{c_{i}}]_{n_{1}} \\\\
                 \rho \vdash e \Downarrow n \quad n < n_{1}
               }
               {
                 \rho \vdash x[e] \Downarrow c_{n}
               }
\\\\
    \inferrule*[lab={\footnotesize{E-Arr}}]
               {
                 \forall i \in \{0 \dots n - 1\}.\:\rho \vdash e_{i} \Downarrow c_{i}
               }
               {
                 \rho \vdash [\overline{e_{i}}]_{n} \Downarrow [\overline{c_{i}}]_{n}
               }
               \quad
    \inferrule*[lab={\footnotesize{E-Inp}}]
               {
               }
               {
                 \rho \vdash \kw{in}_{j} \Downarrow c
               }
  \end{array}
  \]
  \\\\
    \fbox{$\rho \vdash c \Downarrow \rho'; O$}
  \[
  \\
  \begin{array}{c}
    \inferrule*[lab={\footnotesize{E-Decl}}]
               {
                 \rho \vdash e \Downarrow v
               }
               {
                 \rho \vdash \psi\:x = e \Downarrow \rho, x \mapsto v; \cdot
               }
               
               \hspace{0.1cm}

    \inferrule*[lab={\footnotesize{E-LoopT}}]
               {
                 \rho(x) > n
               }
               {
                 \rho \vdash \loops{x}{n}{s} \Downarrow \rho; \cdot
               }

               \\\\
               
    \inferrule*[lab={\footnotesize{E-LoopI}}]
               {
                 \rho(x) \leq n\\\\
                 \rho \vdash s \Downarrow \rho_{1}; O_{1}\\\\
                 \rho_{2} = [\rho_{1}]_{\mathsf{dom}(\rho)}[x \mapsto \rho_{1}(x) + 1]\\\\
                 \rho_{2} \vdash \loops{x}{n}{s} \Downarrow \rho'; O_{2}
               }
               {
                 \rho \vdash \loops{x}{n}{s} \Downarrow \rho'; O_{1}, O_{2}
               }

               \hspace{0.2cm}

    \inferrule*[lab={\footnotesize{E-If}}]
               {
                 \rho \vdash e \Downarrow c\\\\
                 c = \top \Rightarrow s = s_{1}\\\\
                 c = \bot \Rightarrow s = s_{2}\\\\
                 \rho \vdash s \Downarrow \rho'; O
               }
               {
                 \rho \vdash \ite{e}{s_{1}}{s_{2}} \Downarrow \rho'; O
               }

               \\\\
               
    \inferrule*[lab={\footnotesize{E-For}}]
               {
                 \rho, x \mapsto n_{1} \vdash \loops{x}{n_{2}}{s} \Downarrow \rho_{1}; O
               }
               {
                 \rho \vdash \for{x}{n_{1}}{n_{2}}{s} \Downarrow \rho_{1} - \{x\}; O
               }


               \hspace{0.2cm}

    \inferrule*[lab={\footnotesize{E-Out}}]
               {
                 \rho \vdash e \Downarrow c
               }
               {
                 \rho \vdash \kw{out}\:e \Downarrow \rho; c
               }

\end{array}
  \]
\caption{Source semantics (selected rules)}
\label{fig:srcsem}
\end{figure}

\subsubsection*{Source language}Our source language
(Figure~\ref{fig:srclang}) is a simple imperative language. Types
$\psi$ in the language consist of the base types $\sigma$, and arrays
of base types $\sigma[n]$, where $n$ is the array length. Though we
model only one dimensional arrays, our implementation supports higher
dimensional arrays as well. Expressions in the language include the
integer constants $n$, \kw{bool} constants $\top$ and $\bot$,
variables $x$, binary operations $e_{1} + e_{2}$ and $e_{1} > e_{2}$,
conditionals $\cond{e}{e_{1}}{e_{2}}$, array literals 
$[\overline{e_{i}}]_{n}$\footnote{We write $\overline{e}$ (and
  similarly for other symbols) to denote a sequence of expressions.
The length of the sequence is usually clear from the context.}, and
array reads $x[e]$. The expression $\kw{in}_{j}$ denotes input from
party $j$. The statements $s$ in the language comprise of variable
declarations and assignments ($\psi\:x = e$ and $x := e$ resp.),
\kw{for} loops, array writes ($x[e_{1}] := e_{2}$), \kw{if}
statements, and sequence of statements ($s_{1}; s_{2}$). The statement
$\kw{out}\:e$ denotes revealing the value of $e$ to the
parties. The \kw{while} statement is an internal syntax that is not
exposed to the programmer.

The runtime semantics for the source language is shown in
Figure~\ref{fig:srcsem}. Values $v$, runtime environments $\rho$, and
observations $O$ are defined as follows:

\vspace{0.2cm}
$
\small
\begin{array}{rrcl}
    \ftext{Value} & v &::=& c \mid [\overline{c_{i}}]_{n}\\
    \ftext{Runtime environment} & \rho &::=& \cdot \mid \rho[x \mapsto v]\\
    \ftext{Observation} & O & ::= & \cdot \mid c, O \\
\end{array}
$

\vspace{0.2cm}
Values consist of constants and array values. Runtime environment
$\rho$ maps variables to values. Observations are sequences of
constants.

The judgment $\rho \vdash e \Downarrow v$ denotes the big-step
evaluation of an expression~$e$ to a value~$v$ under the runtime
environment~$\rho$. Rule ({\sc{E-Var}}) looks up the value of $x$ in
the environment. Rule ({\sc{E-Add}}) inductively evaluates $e_{1}$ and
$e_{2}$ and evaluates to their addition. Rule ({\sc{E-Read}})
evaluates an array read operation. It first evaluates $x$ to an array
value $[\overline{c_{i}}]_{n_{1}}$, and $e$ to an \kw{uint} value
$n$. It then returns $c_{n}$, the $n$-th index value in the array,
provided $n < n_{1}$, the length of the array. Rule ({\sc{E-Inp}})
evaluates to some constant $c$ denoting party $j$'s input. We model
the inputs to be base constants, an array input can be written in the
language as $[\kw{in}_{j}]_{n}$, which can then evaluate using the
rule ({\sc{E-Arr}}). The remaining rules are straightforward, and are
elided for space reasons.

The judgment $\rho \vdash s \Downarrow \rho'; O$ represents the
big-step evaluation of an statement $s$ under environment $\rho$
producing a new environment $\rho'$ and observations $O$. Rule
({\sc{E-Decl}}) evaluates the expression $e$ to $v$, and returns the
updated environment $\rho[x \mapsto v]$, with empty observations. Rule
({\sc{E-If}}) evaluates the guard expression, and then evaluates
either $s_{1}$ or $s_{2}$ accordingly. $\kw{for}$ statements evaluate
through the internal $\kw{while}$ syntax. Specifically, rule
({\sc{E-For}}) updates $\rho$ with the initial counter value $n_{1}$,
evaluates $\loops{x}{n_{2}}{s}$ to $\rho_{1}; O$, and returns
$\rho_{1} - \{x\}$ (removing $x$ from the environment) and $O$. Rule
({\sc{E-LoopI}}) shows the inductive case for \kw{while}
statements, when $\rho(x) \leq n$. The rule evaluates $s$, producing
$\rho_{1}; O_{1}$. It then restricts $\rho_{1}$ to the domain of
$\rho$ ($[\rho_{1}]_{\mathsf{dom}(\rho)}$) to remove the variables
added by $s$, increments the value if $x$, and evaluates the
\kw{while} statement under this updated environment. Rule
({\sc{E-LoopT}}) is the termination case for \kw{while} statements,
when $\rho(x) > n$. Finally, the rule ({\sc{E-Out}}) evaluates the
expression, and adds it value to the observations.

\newcommand{\lcond}[4]{\ensuremath{{{#2}\:?_{{#1}}\:{#3}\::{#4}}}}
%\newcommand{\for}[4]{\ensuremath{\kw{for}\:{#1}\:\kw{in}\:[{#2}, {#3}]\:\kw{do}\:{#4}}}
%\newcommand{\ite}[3]{\ensuremath{\kw{if}({#1}, {#2}, {#3})}}
%\newcommand{\loops}[3]{\ensuremath{\kw{while}\:{#1} \leq {#2}\:\kw{do}\:{#3}}}

\begin{figure}
  \small
  \[
  \begin{array}{rrcl}
    \ftext{Secret label} & m &::=& \mathcal{A} \mid \mathcal{B}\\
    \ftext{Label} & \ell &::=& \mathcal{P} \mid m\\
    \ftext{Type} & \tau &::=& \sigma^{\ell} \mid \sigma^{\ell}[n]\\
    \ftext{Expression} & \widetilde{e} &::=& c \mid x \mid \widetilde{e_{1}} +_{\ell} \widetilde{e_{2}} \mid \widetilde{e_{1}} >_{\ell} \widetilde{e_{2}} \\
    & & \mid & \lcond{\ell}{{\widetilde{e}}}{{\widetilde{e_{1}}}}{{\widetilde{e_{2}}}} \mid x[\widetilde{e}] \mid [\overline{\widetilde{e_{i}}}]_{n} \mid \kw{in}^{m}_{j} \mid \widetilde{e} \rhd m\\
    \ftext{Statement} & \widetilde{s} &::=& \tau\:x = \widetilde{e} \mid x := \widetilde{e} \mid \dots \mid \widetilde{s_{1}}; \widetilde{s_{2}} \mid \dots\\
  \end{array}
  \]
\caption{Target language}
\label{fig:tgtlang}
\end{figure}

\begin{figure}
  \small
  \[
  \begin{array}{rrcl}
    \ftext{Wire id} & r &&\\
    \ftext{Circuit gate} & \kappa & ::= & r \mid \kw{in}^{m}_{j} \mid \kw{add}\:\kappa_{1}\:\kappa_{2} \mid \kw{gt}\:\kappa_{1}\:\kappa_{2}\\
    & & \mid & \kw{mux}\:\kappa\:\kappa_{1}\:\kappa_{2} \mid \widetilde{w} \rhd m\\
    \ftext{Base value} & \widetilde{w} & ::= & c \mid \kappa\\
    \ftext{Value} & \widetilde{v} & ::= & \widetilde{w} \mid [\overline{\widetilde{w}_{i}}]_{n}\\
    \ftext{Runtime environment} & \widetilde{\rho} & ::= & \cdot \mid \widetilde{\rho}[x \mapsto \widetilde{v}]\\
    \ftext{Circuit} & \chi & ::= & \cdot \mid \kw{bind}\:\kappa\:r \mid \kw{out}\:\kappa \mid \chi_{1}; \chi_{2}
  \end{array}
  \]
\caption{Target runtime}
\label{fig:tgtruntime}
\end{figure}


\begin{figure}
  \small
  \fbox{$\widetilde{\rho} \vdash \widetilde{e} \Downarrow \widetilde{v}$}
  \[
  \\
  \begin{array}{c}
    \inferrule*[lab={\footnotesize{S-Var}}]
               {
               }
               {
                 \widetilde{\rho} \vdash x \Downarrow \widetilde{\rho}(x)
               }
               
    \inferrule*[lab={\footnotesize{S-PAdd}}]
               {
                 \forall i \in \{1, 2\}.\:\widetilde{\rho} \vdash \widetilde{e_{i}} \Downarrow n_{i}
               }
               {
                 \widetilde{\rho} \vdash \widetilde{e_{1}} +_{\mathcal{P}} \widetilde{e_{2}} \Downarrow n_{1} + n_{2}
               }
               
    \inferrule*[lab={\footnotesize{S-Read}}]
               {
                 \widetilde{\rho} \vdash x \Downarrow [\overline{\widetilde{w_{i}}}]_{n_{1}} \\\\
                 \widetilde{\rho} \vdash \widetilde{e} \Downarrow n \quad n < n_{1}
               }
               {
                 \widetilde{\rho} \vdash x[\widetilde{e}] \Downarrow \widetilde{w_{n}}
               }\\\\
    \inferrule*[lab={\footnotesize{S-SAdd}}]
               {
                 \forall i \in \{1, 2\}.\:\widetilde{\rho} \vdash \widetilde{e_{i}} \Downarrow \kappa_{i}
               }
               {
                 \widetilde{\rho} \vdash \widetilde{e_{1}} +_{\mathcal{A}} \widetilde{e_{2}} \Downarrow \kw{add}\:\kappa_{1}\:\kappa_{2}
               }

               \hspace{0.3cm}

    \inferrule*[lab={\footnotesize{S-SGt}}]
               {
                 \forall i \in \{1, 2\}.\:\widetilde{\rho} \vdash \widetilde{e_{i}} \Downarrow \kappa_{i}
               }
               {
                 \widetilde{\rho} \vdash \widetilde{e_{1}} >_{\mathcal{B}} \widetilde{e_{2}} \Downarrow \kw{gt}\:\kappa_{1}\:\kappa_{2}
               }\\\\
               \inferrule*[lab={\footnotesize{S-SCond}}]
               {
                 \forall i \in \{1, 2, 3\}.\:\widetilde{\rho} \vdash \widetilde{e_{i}} \Downarrow \kappa_{i}
               }
               {
                 \widetilde{\rho} \vdash \lcond{\mathcal{B}}{\widetilde{e_{1}}}{\widetilde{e_{2}}}{\widetilde{e_{3}}} \Downarrow \kw{mux}\:\kappa_{1}\:\kappa_{2}\:\kappa_{3}
               }
               \hspace{0.2cm}
    \inferrule*[lab={\footnotesize{S-Coerce}}]
               {
                 \widetilde{\rho} \vdash \widetilde{e} \Downarrow \widetilde{v}
               }
               {
                 \widetilde{\rho} \vdash \widetilde{e} \rhd m \Downarrow \widetilde{v} \rhd m
               }
               \\\\
               %% \inferrule*[lab={\footnotesize{S-Arr}}]
               %% {
               %%   \forall i \in \{0 \dots n - 1\}.\:\widetilde{\rho} \vdash \widetilde{e_{i}} \Downarrow \widetilde{w_{i}}
               %% }
               %% {
               %%   \widetilde{\rho} \vdash [\overline{\widetilde{e_{i}}}]_{n} \Downarrow [\overline{\widetilde{w_{i}}}]_{n}
               %% }

               \inferrule*[lab={\footnotesize{S-PCond}}]
               {
                 \widetilde{\rho} \vdash \widetilde{e} \Downarrow c\\\\
                 c = \top \Rightarrow \widetilde{e'} = \widetilde{e_{1}}\\
                 c = \bot \Rightarrow \widetilde{e'} = \widetilde{e_{2}}\\\\
                 \widetilde{\rho} \vdash \widetilde{e'} \Downarrow \widetilde{v}
               }
               {
                 \widetilde{\rho} \vdash \lcond{\mathcal{P}}{\widetilde{e}}{\widetilde{e_{1}}}{\widetilde{e_{2}}} \Downarrow \widetilde{v}
               }
               %\hspace{0.5cm}
               
    \inferrule*[lab={\footnotesize{S-Inp}}]
               {
               }
               {
                 \widetilde{\rho} \vdash \kw{in}^{m}_{j} \Downarrow \kw{in}^{m}_{j}
               }
  \end{array}
  \]
  \\\\
    \fbox{$\widetilde{\rho} \vdash \widetilde{c} \Downarrow \widetilde{\rho'}; \chi$}
  \[
  \\
  \begin{array}{c}
    \inferrule*[lab={\footnotesize{S-DeclC}}]
               {
                 \widetilde{\rho} \vdash \widetilde{e} \Downarrow \widetilde{v}\\\\
                 \widetilde{v} = c \vee \widetilde{v} = [\overline{c_{i}}]_{n}\\\\
                 \widetilde{\rho'} = \widetilde{\rho}, x \mapsto \widetilde{v}
               }
               {
                 \widetilde{\rho} \vdash \tau\:x = \widetilde{e} \Downarrow \widetilde{\rho'}; \cdot
               }
               
               \hspace{0.5cm}

    \inferrule*[lab={\footnotesize{S-DeclCkt}}]
               {
                 \widetilde{\rho} \vdash \widetilde{e} \Downarrow \kappa\quad
                 \mathsf{fresh}\:r\\\\
                 \widetilde{\rho'} = \widetilde{\rho}, x \mapsto r\quad
                 \chi = \kw{bind}\:\kappa\:r
               }
               {
                 \widetilde{\rho} \vdash \tau\:x = \widetilde{e} \Downarrow \widetilde{\rho'}; \chi
               }

\\\\
    \inferrule*[lab={\footnotesize{S-DeclCktA}}]
               {
                 \widetilde{\rho} \vdash \widetilde{e} \Downarrow [\overline{\kappa_{i}}]_{n}\\\\
                 \forall i \in \{0 \dots n - 1\}.\:\mathsf{fresh}\:r_{i}\\\\
                 \widetilde{\rho'} = \widetilde{\rho}, x \mapsto [r_{i}]_{n}\quad
                 \chi = \overline{\kw{bind}\:\kappa_{i}\:r_{i}}
               }
               {
                 \widetilde{\rho} \vdash \tau\:x = \widetilde{e} \Downarrow \widetilde{\rho'}; \chi
               }

               \hspace{0.2cm}
               
    \inferrule*[lab={\footnotesize{S-Out}}]
               {
                 \widetilde{\rho} \vdash \widetilde{e} \Downarrow \kappa
               }
               {
                 \widetilde{\rho} \vdash \kw{out}\:\widetilde{e} \Downarrow \widetilde{\rho}; \kw{out}\:\kappa
               }

               \\\\
    \inferrule*[lab={\footnotesize{S-If}}]
               {
                 \widetilde{\rho} \vdash \widetilde{e} \Downarrow c\\\\
                 c = \top \Rightarrow \widetilde{s} = \widetilde{s_{1}}\\\\
                 c = \bot \Rightarrow \widetilde{s} = \widetilde{s_{2}}\\\\
                 \widetilde{\rho} \vdash \widetilde{s} \Downarrow \widetilde{\rho'}; \chi
               }
               {
                 \widetilde{\rho} \vdash \ite{\widetilde{e}}{\widetilde{s_{1}}}{\widetilde{s_{2}}} \Downarrow \widetilde{\rho'}; \chi
               }

               \hspace{0.5cm}
               
    \inferrule*[lab={\footnotesize{S-WriteCkt}}]
               {
                 \widetilde{\rho} \vdash x \Downarrow [\overline{r_{i}}]_{n}\quad
                 \widetilde{\rho} \vdash \widetilde{e_{1}} \Downarrow n_{1}\\\\
                 n_{1} < n\quad
                 \mathsf{fresh}\:r\quad
                 \widetilde{\rho} \vdash \widetilde{e_{2}} \Downarrow \kappa\\\\
                 \widetilde{\rho'} = \widetilde{\rho}[x \mapsto [\overline{r_{i}}]_{n}[n_{1} \mapsto r]]
               }
               {
                 \widetilde{\rho} \vdash x[\widetilde{e_{1}}] := \widetilde{e_{2}} \Downarrow \widetilde{\rho'}; \kw{bind}\:\kappa\:r
               }

\end{array}
  \]
\caption{Target semantics (selected rules)}
\label{fig:tgtsem}
\end{figure}

\subsubsection*{Target language} Figure~\ref{fig:tgtlang} shows the
target language of our compiler. The syntax follows that of the source
languag, except that the types and operators are \emph{labeled}. Below
we mainly focus on the bits that are different from the source
language.

Labels $\ell$ consist of secret labels $\mathcal{A}$ and
$\mathcal{B}$, denoting the arithmetic and boolean shared secrets
resp., and the public label $\mathcal{P}$. Types $\tau$ are then
labeled base types $\sigma^{\ell}$ and arrays of labeled base types
$\sigma^{\ell}[n]$.

Most of the expression forms $\widetilde{e}$ are same as $e$, except
that the binary operators, and the conditional forms
are annotated with label $\ell$, denoting how the operators should be
evaluated ($\mathcal{P}$ for in-clear, and $\mathcal{A}$ and
$\mathcal{B}$ for using arithmetic or boolean circuits resp.). The
form $\widetilde{e} \rhd m$ denotes coercing $\widetilde{e}$ to be
$m$-secret shared. The statements $\widetilde{s}$ are analogous to
$s$.

The target semantics models the semantics of the code output by
our compiler (C++ code in our implementation). It computes over the
public parts of the program, emitting a circuit that can later be
evaluated using a cryptographic MPC protocol. Crucially, this phase of
the semantics does not have access to the secrets.

Figure~\ref{fig:tgtruntime} shows the runtime syntax. A wire id range
$r$ denotes a set of circuit wires that carry the value of a secret
shared variable in the output circuit. Circuit gates $\kappa$ are
$r$, input gates, \kw{add}, \kw{gt}, and \kw{mux} gates, and coerce
gates $\widetilde{w} \rhd m$. Target values $\widetilde{v}$ then
consist of base values $c$ and $\kappa$, and arrays of base values.

Figure~\ref{fig:tgtsem} shows the judgments for the target
semantics. We first focus on the expression evaluation judgment
$\widetilde{\rho} \vdash \widetilde{e} \Downarrow
\widetilde{v}$. Rules ({\sc{S-PAdd}}) and ({\sc{S-SAdd}}) illustrate
the significance of the operator labels. In particular, the rule
({\sc{S-PAdd}}) evaluates a public addition $\widetilde{e_{1}}
+_{\mathcal{P}} \widetilde{e_{2}}$ to $n_{1} + n_{2}$, similar to the
source language. In contrast, the rule
({\sc{S-SAdd}}) evaluates a secret addition $\widetilde{e_{1}}
+_{\mathcal{A}} \widetilde{e_{2}}$, using an arithmetic add
gate, to $\kw{add}\:\kappa_{1}\:\kappa_{2}$. Since our compiler
compiles secret add operations to arithmetic gates, the target
expressions from our compiler never have $\widetilde{e_{1}}
+_{\mathcal{B}} \widetilde{e_{2}}$ (we present the compilation rules
later in the section). Rules ({\sc{S-PCond}}) and ({\sc{S-SCond}})
are along the similar lines. Rule ({\sc{S-PCond}}) evaluates a public
conditional to the values from one of the branches, while the rule
({\sc{S-SCond}}) evaluates to a \kw{mux} gate that takes input
circuits from the conditional ($\kappa_{1}$) and both the branches
($\kappa_{2}$ and $\kappa_{3}$). Rules {\sc{S-Coerce}} and
{\sc{S-Inp}} evaluate to the coerce and input gates respectively.

Statement evaluation $\widetilde{\rho} \vdash \widetilde{s} \Downarrow
\widetilde{\rho'}; \chi$ evaluates a target statement to produce a new
environment $\widetilde{\rho'}$, and a circuit $\chi$. $\chi$ (shown
in Figure~\ref{fig:tgtlang}) is either empty, \kw{bind}-ing of a
circuit gate $\kappa$ to range $r$, \kw{out} gate, or a sequence of
circuits. Rules ({\sc{S-DeclC}}), ({\sc{S-DeclCkt}}), and
({\sc{S-DeclCktArr}}) show the variable declaration cases. Rule
({\sc{S-DeclC}}) shows the case when $\widetilde{e}$ evaluates to
$\widetilde{v}$, where $\widetilde{v}$ is a constant or an array of
constants. In this case, the mapping $x \mapsto \widetilde{v}$ is
added to the environment, and the resulting circuit is empty. When
$\widetilde{e}$ evaluates to a circuit $\kappa$, rule
({\sc{S-DeclCkt}}) picks a fresh $r$, adds the mapping $x \mapsto r$
to the environment, and outputs the circuit
$\kw{bind}\:\kappa\:r$. Rule ({\sc{S-DeclCktA}}) is analogous for
$\widetilde{e}$ evaluating to an array of circuits. The variable
assignment rules (not shown in the figure) are similar. Rule
({\sc{S-WriteCkt}}) shows the case for writing to an array, where the
array contents are secret. Finally, rule ({\sc{S-Out}}) compiles to an
\kw{out} circuit.

\begin{figure}
  \small
  \fbox{$\widehat{\rho_{1}}, \widehat{\rho_{2}} \vdash \kappa \Downarrow b_{1}, b_{2}$}
  \[
  \\
  \begin{array}{c}
    %% \inferrule*[lab={\footnotesize{C-Wire}}]
    %%            {
    %%              b_{1}, b_{2} = \widehat{\rho_{1}}(r), \widehat{\rho_{2}}(r)
    %%            }
    %%            {
    %%              \widehat{\rho_{1}}, \widehat{\rho_{2}} \vdash r \Downarrow b_{1}, b_{2}
    %%            }               
    \inferrule*[lab={\footnotesize{C-In}}]
               {
                 b_{1}, b_{2} = \mathcal{E}_{m}(c)
               }
               {
                 \widehat{\rho_{1}}, \widehat{\rho_{2}} \vdash \kw{in}^{m}_{j} \Downarrow b_{1}, b_{2}
               }

               \hspace{0.3cm}
               
    \inferrule*[lab={\footnotesize{C-Coerce}}]
               {
                 \widehat{\rho_{1}}, \widehat{\rho_{2}} \vdash \kappa \Downarrow b_{1}, b_{2}\\\\
                 c = \mathcal{D}_{m_{1}}(b_{1}, b_{2})\\\\
                 b'_{1}, b'_{2} = \mathcal{E}_{m}(c)
               }
               {
                 \widehat{\rho_{1}}, \widehat{\rho_{2}} \vdash \kappa \rhd m \Downarrow b'_{1}, b'_{2}
               }
\\\\

    \inferrule*[lab={\footnotesize{C-Add}}]
               {
                 \forall i \in \{1, 2\}.\:\widehat{\rho_{1}}, \widehat{\rho_{2}} \vdash \kappa_{i} \Downarrow b_{1i}, b_{2i}\quad
                 n_{i} = \mathcal{D}_{\mathcal{A}}(b_{1i}, b_{2i})\\\\
                 b_{1}, b_{2} = \mathcal{E}_{\mathcal{A}}(n_{1} + n_{2})
               }
               {
                 \widehat{\rho_{1}}, \widehat{\rho_{2}} \vdash \kw{add}\:\kappa_{1}\:\kappa_{2} \Downarrow b_{1}, b_{2}
               }

\\\\

    \inferrule*[lab={\footnotesize{C-Gt}}]
               {
                 \forall i \in \{1, 2\}.\:\widehat{\rho_{1}}, \widehat{\rho_{2}} \vdash \kappa_{i} \Downarrow b_{1i}, b_{2i}\quad
                 n_{i} = \mathcal{D}_{\mathcal{B}}(b_{1i}, b_{2i})\\\\
                 b_{1}, b_{2} = \mathcal{E}_{\mathcal{B}}(n_{1} > n_{2})
               }
               {
                 \widehat{\rho_{1}}, \widehat{\rho_{2}} \vdash \kw{gt}\:\kappa_{1}\:\kappa_{2} \Downarrow b_{1}, b_{2}
               }

\\\\

    \inferrule*[lab={\footnotesize{C-Mux}}]
               {
                 \forall i \in \{1, 2, 3\}.\:\widehat{\rho_{1}}, \widehat{\rho_{2}} \vdash \kappa_{i} \Downarrow b_{1i}, b_{2i}\quad
                 c_{i} = \mathcal{D}_{\mathcal{B}}(b_{1i}, b_{2i})\\\\
                 c_{1} = \top \Rightarrow b_{1}, b_{2} = \mathcal{E}_{\mathcal{B}}(c_{2})\quad
                 c_{1} = \bot \Rightarrow b_{1}, b_{2} = \mathcal{E}_{\mathcal{B}}(c_{3})
               }
               {
                 \widehat{\rho_{1}}, \widehat{\rho_{2}} \vdash \kw{mux}\:\kappa_{1}\:\kappa_{2}\:\kappa_{3} \Downarrow b_{1}, b_{2}
               }

  \end{array}
  \]
  \\\\
    \fbox{$\widehat{\rho_{1}}, \widehat{\rho_{2}} \vdash \chi \Downarrow \widehat{\rho'_{1}}, \widehat{\rho'_{2}}; O$}
  \[
  \\
  \begin{array}{c}
    \inferrule*[lab={\footnotesize{C-Bind}}]
               {
                 \widehat{\rho_{1}}, \widehat{\rho_{2}} \vdash \kappa \Downarrow b_{1}, b_{2}\\\\
                 \widehat{\rho'_{1}} = \widehat{\rho_{1}}[r \mapsto b_{1}] \quad
                 \widehat{\rho'_{2}} = \widehat{\rho_{2}}[r \mapsto b_{2}]
               }
               {
                 \widehat{\rho_{1}}, \widehat{\rho_{2}} \vdash \kw{bind}\:\kappa\:r \Downarrow \widehat{\rho'_{1}}, \widehat{\rho'_{2}}; \cdot
               }
               
               \hspace{0.1cm}

    \inferrule*[lab={\footnotesize{C-Out}}]
               {
                 \widehat{\rho_{1}}, \widehat{\rho_{2}} \vdash \kappa \Downarrow b_{1}, b_{2}\\\\
                 c = \mathcal{D}_{m}(b_{1}, b_{2})
               }
               {
                 \widehat{\rho_{1}}, \widehat{\rho_{2}} \vdash \kw{out}\:\kappa \Downarrow \widehat{\rho_{1}}, \widehat{\rho_{2}}; c
               }
\end{array}
  \]
\caption{Circuit semantics (selected rules)}
\label{fig:cktsem}
\end{figure}


\subsubsection*{Circuit semantics} The target semantics produces a
circuit to compute over the secret data using an MPC protocol. With
our circuit semantics, we model the \emph{functional} aspect of the
protocol, parametrized by the sharing functions.

During the circuit evaluation, the wire ranges $r$ are
mapped to (opaque) strings $b$. The semantics of these strings is
given by pairs of encryption-decryption functions, written as
$\mathcal{E}_{m}$ and $\mathcal{D}_{m}$ (where $m$ is either
$\mathcal{A}$ or $\mathcal{B}$). More concretely,
$\mathcal{E}_{m}(c)$ returns a pair of two string $b_{1}$ and $b_{2}$
(shares of the two parties), with the property that
$\mathcal{D}_{m}(b_{1}, b_{2}) = c$. We assume that the underlying MPC
protocol instantiates $\mathcal{E}_{m}$ and $\mathcal{D}_{m}$
appropriately.

The circuit semantics is shown in Figure~\ref{fig:cktsem} using the
judgments $\widehat{\rho_{1}}, \widehat{\rho_{1}} \vdash \kappa
\Downarrow b_{1}, b_{2}$, and $\widehat{\rho_{1}}, \widehat{\rho_{2}}
\vdash \chi \Downarrow \widehat{\rho_{1}'}, \widehat{\rho_{2}'}; O$,
where $\widehat{\rho_{1}}$ and $\widehat{\rho_{2}}$ are the circuit
environments of the two parties, mapping wire ranges $r$ to strings
$b$. Rule {\sc{C-Add}} evinces the pattern for evaluating circuit
gates $\kappa$. To evaluate $\kw{add}\:\kappa_{1}\:\kappa_{2}$, the
rule first evaluates $\kappa_{1}$ to $(b_{11}, b_{21})$ and
$\kappa_{2}$ to $(b_{12}, b_{22})$. Shares $(b_{11}, b_{21})$ are then
combined using $\mathcal{D_{\mathcal{A}}}$ to $n_{1}$, and similarly
$(b_{12}, b_{22})$ are combined to $n_{2}$. The final output of the
\kw{add} gate is then $\mathcal{E}_{\mathcal{A}}(n_{1} + n_{2})$. Note
that this is a functional description of how the \kw{add} gate
evaluates, of course, concretely $n_{1}$ and $n_{2}$ are not observed
by the parties. Rule {\sc{C-Coerce}} re-encrypts the shares using the
scheme for $m$. Coming to the evaluation of circuits, the evaluation
of \kw{bind} updates the mapping of $r$ in the input environments, and
the rule {\sc{C-Out}} outputs the clear value $c$ to the observations.

\subsubsection*{Compilation judgments}

\begin{theorem}[Correctness of the compiler]
  
\end{theorem}


\begin{figure}
  \small
  \fbox{$\Gamma \vdash e : \tau \leadsto \widetilde{e}$}
  \[
  \begin{array}{c}
     \inferrule*[lab={\footnotesize{T-Cons}}]
               {
                 \tau = \mathsf{typeof}(c)^{\mathcal{P}}
               }
               {
                 \Gamma \vdash c : \tau \leadsto c 
               }

     \inferrule*[lab={\footnotesize{T-Add}}]
               {
                 \forall i \in \{1,2\}.\:\Gamma \vdash e_{i} : \kw{uint}^{\ell} \leadsto \widetilde{e_{i}}\\\\
                 \ell = \mathcal{P} \vee \ell = \mathcal{A}
               }
               {
                 \Gamma \vdash e_{1} + e_{2} : \kw{uint}^{\ell} \leadsto \widetilde{e_{1}} +_{\ell} \widetilde{e_{2}}
               }

    %% \inferrule*[lab={\footnotesize{T-Cons}}]
    %%            {
    %%              \tau = \mathsf{typeof}(c)^{\mathcal{P}}
    %%            }
    %%            {
    %%              \Gamma \vdash c : \tau \leadsto c
    %%            }

    %%  \inferrule*[lab={\footnotesize{T-Var}}]
    %%            {
    %%            }
    %%            {
    %%              \Gamma \vdash x : \Gamma(x) \leadsto x
    %%            }
\\\\
     \inferrule*[lab={\footnotesize{T-Gt}}]
               {
                 \forall i \in \{1,2\}.\:\Gamma \vdash e_{i} : \kw{uint}^{\ell} \leadsto \widetilde{e_{i}}\\\\
                 \ell = \mathcal{P} \vee \ell = \mathcal{B}
               }
               {
                 \Gamma \vdash e_{1} > e_{2} : \kw{bool}^{\ell} \leadsto \widetilde{e_{1}} >_{\ell} \widetilde{e_{2}}
               }

     \inferrule*[lab={\footnotesize{T-Read}}]
               {
                 \Gamma \vdash x : \sigma^{\ell}[n] \leadsto x\\\\
                 \Gamma \vdash e : \kw{uint}^{\mathcal{P}} \leadsto \widetilde{e}\\\\
                 \Gamma \models e < n
               }
               {
                 \Gamma \vdash x[e] : \sigma^{\ell} \leadsto x[\widetilde{e}]
               }

\\\\               

     \inferrule*[lab={\footnotesize{T-Cond}}]
               {
                 \Gamma \vdash e : \kw{bool}^{\ell} \leadsto \widetilde{e}\\\\
                 \forall i \in \{1,2\}.\:\Gamma \vdash e_{i} : \sigma^{\ell'} \leadsto \widetilde{e_{i}}\\\\
                 \ell = \mathcal{P} \vee (\ell = \mathcal{B} \wedge \ell' =\mathcal{B})
               }
               {
                 \Gamma \vdash \cond{e}{e_{1}}{e_{2}} : \sigma^{\ell'} \leadsto \lcond{\ell}{\widetilde{e}}{\widetilde{e_{1}}}{\widetilde{e_{2}}}
               }

     \inferrule*[lab={\footnotesize{T-Inp}}]
               {
               }
               {
                 \Gamma \vdash \kw{in}_{j} : \sigma^{m} \leadsto \kw{in}^{\sigma}_{j}
               }
               
\\\\               

     \inferrule*[lab={\footnotesize{T-Arr}}]
               {
                 \forall i \in \{0 \dots n - 1\}.\:\Gamma \vdash e_{i} : \sigma^{\ell} \leadsto \widetilde{e_{i}}
               }
               {
                 \Gamma \vdash [\overline{e_{i}}]_{n} : \sigma^{\ell}[n] \leadsto [\overline{\widetilde{e_{i}}}]_{n}
               }

     \inferrule*[lab={\footnotesize{T-Sub}}]
               {
                 \Gamma \vdash e : \sigma^{\ell} \leadsto \widetilde{e}
               }
               {
                 \Gamma \vdash e : \sigma^{m} \leadsto \widetilde{e} \rhd m
               }

  \end{array}
  \]
  \\\\
    \fbox{$\Gamma \vdash s \leadsto \widetilde{s} \mid \Gamma'$}
  \[
  \\
  \begin{array}{c}
     \inferrule*[lab={\footnotesize{T-Decl}}]
               {
                 \psi = \sigma \Rightarrow \tau = \sigma^{\ell}\\\\
                 \psi = \sigma[n] \Rightarrow \tau = \sigma^{\ell}[n]\\\\
                 \Gamma \vdash e : \tau \leadsto \widetilde{e}
               }
               {
                 \Gamma \vdash \psi\:x = e \leadsto \tau\:x = \widetilde{e} \mid \Gamma, x:\tau
               }

     \inferrule*[lab={\footnotesize{T-Assgn}}]
               {
                 \Gamma(x) = \sigma^{\ell}\\\\
                 \Gamma \vdash e : \sigma^{\ell} \leadsto \widetilde{e}
               }
               {
                 \Gamma \vdash x := e \leadsto x = \widetilde{e} \mid \Gamma
               }

\\\\

     \inferrule*[lab={\footnotesize{T-For}}]
               {
                 \Gamma' = \Gamma, x::\kw{uint}^{\mathcal{P}}\\\\
                 \Gamma' \vdash \loops{x}{n_{2}}{s} \leadsto \loops{x}{n_{2}}{\widetilde{s}} \mid \Gamma'
               }
               {
                 \Gamma \vdash \for{x}{n_{1}}{n_{2}}{s} \leadsto \for{x}{n_{1}}{n_{2}}{\widetilde{s}} \mid \Gamma
               }

               \\\\

     \inferrule*[lab={\footnotesize{T-Write}}]
               {
                 \Gamma \vdash x : \sigma^{\ell}[n] \leadsto x\\\\
                 \Gamma \vdash e_{1} : \kw{uint}^{\mathcal{P}} \leadsto \widetilde{e_{1}}\\\\
                 \Gamma \vdash e_{2} : \sigma^{\ell} \leadsto \widetilde{e_{2}}\\\\
                 \Gamma \models e_{1} < n
               }
               {
                 \Gamma \vdash x[e_{1}] := e_{2} \leadsto x[\widetilde{e_{1}}] := \widetilde{e_{2}} \mid \Gamma
               }

     \inferrule*[lab={\footnotesize{T-Out}}]
               {
                 \Gamma \vdash e : \sigma^{m} \leadsto \widetilde{e}
               }
               {
                 \Gamma \vdash \kw{out}\:e \leadsto \kw{out}\:\widetilde{e} \mid \Gamma
               }

\\\\

     \inferrule*[lab={\footnotesize{T-If}}]
               {
                 \Gamma \vdash e : \kw{bool}^{\mathcal{P}} \leadsto \widetilde{e}\\\\
                 \forall i \in \{1, 2\}.\:\Gamma \vdash s_{i} \leadsto \widetilde{s_{i}} \mid \_
               }
               {
                 \Gamma \vdash \ite{e}{s_{1}}{s_{2}} \leadsto \ite{\widetilde{e}}{\widetilde{s_{1}}}{\widetilde{s_{2}}}  \mid \Gamma
               }

     \inferrule*[lab={\footnotesize{T-Seq}}]
               {
                 \Gamma \vdash s_{1} \leadsto \widetilde{s_{1}} \mid \Gamma_{1}\\\\
                 \Gamma_{1} \vdash s_{2} \leadsto \widetilde{s_{2}} \mid \Gamma_{2}
               }
               {
                 \Gamma \vdash s_{1}; s_{2} \leadsto \widetilde{s_{1}}; \widetilde{s_{2}} \mid \Gamma_{2}
               }

\\\\

     \inferrule*[lab={\footnotesize{T-While}}]
               {                 
                 \Gamma(x) = \kw{uint}^{\mathcal{P}}\quad
                 \Gamma \vdash s \leadsto \widetilde{s} \mid \_\quad
                 x \notin \mathsf{modifies}(s)
               }
               {
                 \Gamma \vdash \loops{x}{n_{2}}{s} \leadsto \loops{x}{n_{2}}{\widetilde{s}} \mid \Gamma
               }

  \end{array}
  \]
\label{fig:compile}
\caption{Compilation judgments (selected rules)}
\end{figure}

  %% \\
  %% \[
  %% \begin{array}{rrcl}
  %%   \ftext{Value} & v &::=& c \mid [\overline{c_{i}}]_{n}\\
  %%   \ftext{Observation} & O &::=& \cdot \mid c; O\\
  %% \end{array}
  %% \]
