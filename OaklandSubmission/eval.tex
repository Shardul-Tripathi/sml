\section{Evaluation}
\label{sec:eval}
We evaluate \tool on the problem of {\it secure prediction}, where one
party (the server) has a machine learning model, and the other party
(the client) has an input. The goal is to compute the output of the
model on client's input, with the guarantee that the server
learns nothing about the input, and the client learns nothing
about the model beyond what is revealed from the output.

We first implement the benchmarks from Bost et al.~\cite{shafindss}
and \minion~\cite{minionn} (both of which study the same setting),
and show that the performance of the high-level code written in \tool
is comparable to their hand-crafted circuits
(Section~\ref{sec:shallow} and Section~\ref{sec:dnneval}).
%
Next, we demonstrate the generality and programmability aspects of
\tool by implementing state-of-the-art machine learning models from
Tensorflow~\cite{tensorflow} and \bonsai~\cite{bonsai}. Indeed, we
provide the first 2PC implementation of \bonsai.
%
Finally, we
implement a Deep Neural Network (DNN) for CIFAR-10
dataset~\cite{cifar} from \minion~\cite{minionn} to illustrate the use
of partitioning in \tool.

We present the numbers for two network settings, a LAN setting and
a cross-continent WAN setting. The round trip time between the server
and the client machines in the two settings is 1ms and 40ms resp. Each
machine has an Intel(R) Xeon(R) CPU E5-2673 v3 processor running at
2.40GHz with 28 GBs of RAM.
%
%% The  \tool compiler is written in Python and
%% compiles each of our benchmarks in under a second to C++ code that makes calls to the ABY
%% library~\cite{aby}. We use an off-the-shelf solver
%% (SeaHorn~\cite{seahorn}) to check the verification conditions required
%% to prove that the array indices remain in bound. The solver takes less
%% than a minute on our largest benchmark.
Since most of our benchmarks
are related to machine learning, we set up some notation (largely
standard in machine learning) and describe our benchmarks next.

\subsection{Benchmarks}
%\divya{notation can be moved somewhere... } \\

\noindent(We use $[N]$ to denote $\{0,1,\dotsc, N-1\}$. 
Further, given a vector $x\in\R^d$, we say $\mathsf{argmax}\ x = i$ if
$x_i = \mathsf{max}\ \{x_1,\ldots,x_d\}$.)

We focus on the machine learning models for {\it classification}.
A classifier $C$ uses a trained model to {\it predict} a
label $\ell$ for an input data point $x$. For example, given a
data point which is a tuple of height and weight of an individual,
a classifier can predict a label ``male'' or ``female''. The {\it
  accuracy} of a classifier refers to the fraction of data points that
the classifier labels correctly from a given set of test data
points.

\subsubsection*{Standard classifiers}
A binary linear classifier is one of the simplest classifiers. Here,
the input is a data point $x\in\R^d$,
and the model is a vector $w\in\R^d$. The possible labels are
$\ell\in\{\mathit{true},\mathit{false}\}$ and the classifier is
$C_w\equiv w^Tx>0$, where $\cdot^T$ denotes the matrix transpose
operator. This classifier requires $d$ multiplications, $d-1$
additions, and a comparison.

A more interesting classifier is Na\"{i}ve Bayes that predicts labels
from the set $[n]$.
Here, the input data point is a {\it feature}
vector $x=(x_0,x_2,\ldots,x_{d-1})^T$ where each $x_j\in [F]$.
The model has two matrices: a vector $P$ of length $n$ s.t. for each
$i
\in [n]$, $P(i)$ is the log-probability that the output label is $i$.
%$P(i) \equiv \log p(\ell=i)$, where $p(\ell=i)$ is the likelihood
%that the output label is $i$ for $i\in[n]$.
The other matrix $T$ has size $d\times F\times n$ and for each $j\in
[d], k\in [F], i\in [n]$, the entry $T(j)(k)(i)$ is 
%\equiv\log p(x_j|\ell=i)$, i.e.,
the log-probability that  the $j^{th}$ input feature is $k$  conditioned on  the output label being $i$.
The classifier $C_{P,T}(x)$ outputs
\[
{\sf argmax}\ P+\sum_{j=0}^{d-1} T(j)(x_j)
\]
This classifier requires $ndF$ comparisons and additions. \divya{Note
  that computing $T(i)(j)(x_j)$ requires a secret look-up in matrix
  $T$  based on secret input $x_j$ resulting in $F$ comparisons (see
  \sectionref{}).} \cmmt{Is there a +1 etc missing or is this exact?}

A decision tree of size $N$ takes as input an $x\in\R^d$, and the
model consists of a binary tree of depth $d$ with a boolean predicate
assigned to each internal node. The root note is the $0^{th}$ node and
for an internal node $j \in [N]$, the children nodes are $2j+1$ and
$2j+2$. Each internal node $j\in[N]$ at depth $i$ has a predicate
$b_{j}^i\equiv x_i\leq w_{j}$. We start evaluating the tree from the
root and if the predicate at the current node $b_j^i$ is false (resp.,
true) then $x$ is passed to the left (resp., right) child with
predicate $b_{2j+1}^{i+1}$ (resp., $b_{2j+2}^{i+1}$). This process is
repeated till we reach a leaf. The leaves of the tree are labels and
the output label is the leaf visited by this traversal.
Such a (binary) decision tree can be encoded as a polynomial (of
degree linear in depth of the tree) and then the prediction task
reduces to polynomial evaluation. E.g., a binary tree with depth one
and size $N=3$ can be encoded as the polynomial
$(1-b(x))\ell_0 + b(x)\ell_1$ where $b(x)$ is the predicate at the
root, $\ell_0$ is the label of the left leaf and $\ell_1$ is the label
of the right leaf.

\subsubsection*{Deep neural nets}
The next class of classifiers that we benchmark are deep neural nets
or DNNs. A DNN has multiple layers such that each layer computes a
matrix
multiplication followed by an {\it activation} function $f$. The most
common activation functions are square $f(x)=x^2$ and rectifier linear
unit (ReLU) $f(x)=\mathsf{max}(x,0)$.
If $A$ is a matrix then the output of $f(A)$ is also a matrix that is
obtained by applying $f$ to each entry of $A$ pointwise.
Given an input vector $x$, the predicted label of a DNN is
\[
 \mathsf{argmax}\ W_N\cdot f_{N-1}(\ldots f_1(W_1\cdot x)\ldots)
\]
Here, $f_i$'s are the (public) activation functions, the model
consists of matrices $W_i$,  $x \in \R^d$ is the input vector, and the
operator $\cdot$ denotes a matrix multiplication.
Neural nets usually have one or more fully connected layers, each of
which multiplies a matrix with a vector.
Some neural nets have convolution layers and such DNNs are also called
Convolutional Neural Nets or CNNs.
For the purpose of this paper, a convolution can be considered as a
(heavy) matrix-matrix multiplication. The size of matrices manipulated
by a convolution layer grows linearly with {\it window size}
(typically 9 or 25), the number of {\it output channels} (typically
16, 32, or 64), and the size of the matrix input to this layer.
Therefore, fully connected layers are lighter computation-wise compared
to convolution layers. However, the model size of fully connected
layers is larger than those of convolution layers.
In general, DNNs are computationaly heavy but provide
much better accuracies on computer vision tasks than the classifiers
discussed above.

\subsubsection*{State-of-the-art classifiers}
Finally, there are a class of machine learning classifiers that are
much more efficient than
DNNs and provide reasonably good accuracies on standard learning
tasks. \bonsai~\cite{bonsai} is a state-of-the art classifier in this
class and \tool provides the first 2PC protocol for it.
\bonsai takes as input $x \in \R^d$, and its model consists of a
binary tree with $N$ nodes, and a matrix $Z$. Each node $j$ contains
matrices $W_j$ and $V_j$, and a vector $\theta_{j}$. The internal node
$j$  evaluates a predicate $\theta_j^TZx > 0$ to decide whether
to pass $x$ to the left child $2j+1$ or the right child $2j+2$.
The predicted value is

\aseem{I had a hard time parsing this. (1) Should we use $\cdot$ for
  matrix multiplication as we did for DNNs? (2) Precedence of $\circ$
  over scalar multiplication (or is it associative so doesn't matter?)
(3) The argument to $f$ is a scalar (right?), and so is its output,
so then is $\circ$ defined OR is it applying $f$ pointwise to a vector?}

\[
{\sf argmax} \sum_{j = 0}^{N - 1} I_j(x)W_j^TZx\circ f(V_j^TZx) 
\]

Here, $I_j(x)$ is 1 if the $j^{th}$ node is present on the path traversed by $x$
and is zero otherwise. 
The operation $\circ$ is a pointwise dot product, $W_j$'s and $V_j$'s
are matrices. The activation function $f$ is given by $f(y) = y$ if
$-1 < y < 1$ and $\mathrm{sign}(y)$ otherwise. \bonsai can be
seen as a variant of decision trees where the prediction is a function
of the path traversed from root to the leaf and not just the leaf
itself.

In the following, we implement these classifiers in \tool and report
the time taken for making secure predictions. Ideally, the machine
learning classifiers are mathematical expressions over $\R$ that are
usually approximated by floating-point operations. 
As is standard, we port the classifers to integer manipulating
programs by scaling the models and rounding~\cite{minionn}. These
ported classifiers are then implemented in \tool.

\subsection{Standard classifiers}
\label{sec:shallow}
We train the three standard classifiers, linear, Na\"{i}ve Bayes, and
decision trees, on the following data sets from the UCI
machine learning repository~\cite{uci}:
 the Wisconsin Breast Cancer data set, 
Credit Approval data set, Audiology (Standardized) data set, Nursery
data set, and ECG (electrocardiogram) classification data
from~\cite{barni}.

\subsubsection*{Linear classifier}The results for linear
classification are in Table~\ref{tab:lc}. The
input and model are both vectors of length $d$. The time in columns
``Prev. time'' and communication ``Prev. comm'' is the time and total
network communication reported by Bost et al.~\cite{shafindss}. They
report these numbers over a simulated network with 40ms
round trip
time, which is same as our WAN setting. The total execution time
of \tool generated code in the LAN and the WAN setting is reported
next, followed by the total communication.
We observe that the \tool code performance matches that of Bost
et al. On the other hand, the programmer effort in \tool is just 20
lines (last column in the table) of code in the source language, as
opposed to writing low-level circuits. \aseem{They write circuits,
  right?} The execution time is dominated by multiplications and
boolean-ands.
The number of boolean-and gates (\#And) remains constant for both the
benchmarks as they reflect the number of and gates required for a
single 32-bit integer comparison with zero, while the number of
multiplication gates (\#Mul) is the same as the model size $d$.
%% The
%% last column shows the lines of code of \tool program, which is also
%% independent of the model size.

\subsubsection*{Na\"{i}ve Bayes} The results for Na\"{i}ve Bayes are
in Table~\ref{tab:nb}. As before, $d$ denotes the length of the input
feature vector $x$ and $F$ is the number of possible values of $x_i$
for each $i \in [d]$.
Although, there are no multiplications in these benchmarks, these do have a significant number of comparisons (due to secret look-up) that result in many boolean-and gates (13k/750k denote 13000/75000). 
Here, again columns 4 and 5 show the results reported in~\cite{shafindss}. We observe that \tool generated code has competitive performance on this task. This improvement is significant as~\cite{shafindss} uses custom designed protocols and we are executing a generic two party computation (2PC). Moreover,~\cite{shafindss} remarks that generic 2PC did not scale to the smallest of their Na\"{i}ve Bayes classifiers and they had to scale down the prediction task and there a Yao-based implementation was 500 times slower. We have shown that by using \tool, generic 2PC can scale to real prediction tasks and provide performance competitive with specialized protocols. 

This claim is further validated by Table~\ref{tab:dt}.
In~\cite{wu}, Wu et al. describe specialized protocols for trees and forests (that perform better than~\cite{shafindss}) and we report their performance in columns 3 and 4 of Table~\ref{tab:dt}.
Similar to~\cite{shafindss},~\cite{wu} remarks that these computations are infeasible with Yao-based 2PC.
Again, we observe that the performance of \tool generated generic 2PC code is competitive with specialized protocols. Therefore, we believe that \tool generated code can be a better baseline than Yao-based 2PC when comparing against generic 2PC-based implementations for secure prediction tasks.

\begin{table*}
\begin{tabular}{c|c|c|c|c |c|c|c|c|c|c}
Dataset & $d$  & Prev. time (s) & Prev. comm (kb) & LAN time (s) & WAN time (s) & Comm. (kb)  & \#And & \#Mul & \#Gates & LOC\\
\hline
Breast cancer & 30 & 0.3 & 36 & 0.1 & 0.3 & 25 & 95 & 30 & 727 & 20\\
\hline
Credit & 47 & 0.3 & 41 & 0.1 & 0.3 & 36 & 95 & 47 & 795 & 20\\
\hline
\end{tabular}

 \caption{Linear classification results. We compare our results (columns 5, 6, 7) with~\cite{shafindss} (columns 3 and 4)}
 \label{tab:lc} 
\end{table*}

\begin{table*}
\begin{tabular}{c|c|c|c|c|c |c|c|c|c|c|c}
Dataset & $d$ & $F$ & Prev. time (s) & Prev. Comm (Mb) & LAN time (s) & WAN time (s) & Comm. (Mb)  & \#And & \#Mul & \#Gates  & LOC\\
\hline
Nursery & 5 & 9 & 1.5 & 0.2 & 0.1 & 0.4 & 0.6 & 13k & 0 & 73k & 50\\
\hline
Audiology & 24 & 70 & 3.9 & 2.0 & 1.5 & 2.9 & 37 & 750k & 0 & 4219k & 50\\
\hline
\end{tabular}

 \caption{Na\"{i}ve Bayes results. We compare our results (columns 6 , 7, 8) with~\cite{shafindss} (columns 4 and 5)}
 \label{tab:nb} 
\end{table*}

\begin{table*}
\begin{tabular}{c|c|c|c |c|c|c|c|c|c|c}
Dataset  & $N$ & Prev. time (s) & Prev. Comm (kb) & LAN time (s) & WAN time (s) & Comm. (kb)  & \#And & \#Mul & \#Gates & LOC\\
\hline
Nursery & 4 & 0.3 & 102 & 0.1 & 0.3 & 32 & 504 & 3 & 3324 & 20\\
\hline
ECG &  6 & 0.4 & 102 & 0.1 & 0.4 & 49 & 756 & 5 & 5002 & 20\\
\hline
\end{tabular}

 \caption{Decision tree benchmarks. We compare our results (columns 5 , 6, 7) with~\cite{wu} (columns 3 and 4)}
 \label{tab:dt} 
\end{table*}


%dataset  | size | gates | and | mul | depth | length

%dataset | time | comm | timeL | timeW | comm
\subsection{Deep neural nets}
\label{sec:dnneval}
We evaluate \tool on the DNNs described in \minion~\cite{minionn} and the results are reported in Table~\ref{tab:nn}\footnote{\minion does not report the network round-trip time~\cite{minionn}.}.
Our goal here is to demonstrate that \tool
can provide performance competitive to specialized protocols for DNNs described in~\cite{secureml,cryptonets,minionn}. 
The first benchmark  is the DNN described in SecureML~\cite{secureml}.
It has three fully connected layers with square as the activation function.
This DNN is light-weight as it requires only arithmetic computations (additions and multiplications)
except for the {\it argmax} at the end.
Next, we implement the DNN described in Cryptonets~\cite{cryptonets} in \tool.
This DNN also uses square as the activation function and has one convolution (with 5 output channels) and one fully connected layer.
Finally, we compare against \minion on a CNN
with two convolutions (with 16 output channels each) and two fully connected layers.
In contrast to the previous two DNNs, this DNN uses ReLU for activation and
has significantly higher number of boolean-and gates.
Recall, that square activation can be implemented entirely using arithmetic gates but ReLU requires boolean-and gates.  For a complete description of these benchmarks and their accuracies,  the reader is referred to the original references.

For the benchmarks in Table~\ref{tab:nn}, \minion
outperforms SecureML and Cryptonets and columns 2 and 3 show the time taken by \minion. 
The column ``Model size'' is the number of 32 bit integers in the trained model.
We observe that our performance is competitive with \minion in the LAN and the WAN settings and the lines of code required are still small.
We note that \minion and \tool implementations are both based on ABY.
These results are surprising as \minion has a much more efficient preprocessing phase and has SIMD
(single instruction multiple data) capabilities that allow \minion to perform
matrix operations efficiently. Our language and compiler 
focus on generic 2PC and such optimizations are beyond the scope of this paper.
However, \minion also reports performance results on a bigger DNN with 7 convolution layers.
This benchmark requires pipelining and we discuss it in Section~\ref{sec:pipeeval}.
%net | minionT | minionC | timeL | timeW | comm | gates | and | mul | length 

\begin{table*}
\begin{tabular}{c|c|c|c |c|c|c|c|c|c|c}
DNN  & Prev. time (s) & Prev. Comm (Mb) & LAN time (s) & WAN time (s) & Comm. (Mb)  & \#And & \#Mul & \#Gates & Model Size & LOC\\
\hline
SecureML   &  1.1 & 15.8 & 0.7 & 1.7  & 76   &  2k   & 119k & 366k   & 119k & 78\\
\hline
Cryptonets &  1.3 & 47.6 & 0.6 & 1.6  & 70    & 2k    & 108k & 316k & 86k & 88\\
\hline
CNN        &  9.4 & 657.5& 5.1 & 11.6 & 501  & 1640k & 667k & 9480k & 35k & 154\\
\hline
\end{tabular}

 \caption{DNN benchmarks. We compare our results (columns 4 , 5, 6) with~\cite{minionn} (columns 2 and 3)}
 \label{tab:nn} 
\end{table*}


\subsection{Practical models}
Tensorflow~\cite{tensorflow} is a standard machine learning toolkit.
Its introductory tutorial describes two prediction models for handwritten digit recognition
using the MNIST dataset~\cite{mnist}.
Each image in this dataset is a greyscale $28\times 28$ image of digits 0 to 9.
The first model that the tutorial describes is softmax regression
that provides an accuracy of 92\%. The classifier here evaluates
\[
 \mathsf{argmax}\ Wx+b
\]
Here, $i\in [10]$, $x$ is a 784 length vector obtained from the input image,
each $W$ is a $10\times 784$ matrix, and each $b$ is a $784$ length vector. 
We implement this classifer in \tool and the results are shown in the first row of Table~\ref{tab:tf}.
Since, we are not familiar with any other tools that have used this model as a benchmark,
we only report execution time of \tool generated code.


The next classifer in the Tensorflow tutorial is a convolution neural net with two convolutions
(with 32 output channels) and two fully connected layers with ReLU as the activation function.
This DNN is both bigger and more accurate than the DNNs presented in the previous section.
In particular, it has an accuracy of 99.2\%\footnote{The accuracy of the integer model is 0.04\% higher than the floating-point model.}.
We observe that this DNN can take a minute per prediction in the WAN setting and is the largest
benchmark that we have evaluated without pipelining.


% benchmark | size | gates | and | mul | depth  | length | comm | timeL | timeW

\begin{table*}
\begin{tabular}{c|c|c|c |c|c|c|c|c|c | c}
Classifier       & LAN time (s) & WAN time (s) & Comm. (Mb)  & \#And & \#Mul & \#Gates & Model size & LOC\\
\hline
Regression &  0.1         & 0.7         & 5            & 2k    & 8k    &  35k    & 8k   & 38\\
\hline
CNN        &  30.5        & 60.3        & 2955         & 6082k & 4163k &  42104k & 3226k& 172\\
\hline
\end{tabular}

 \caption{Tensorflow tutorial benchmarks}
 \label{tab:tf} 
\end{table*}


Before discussing pipelining, we show \bonsai results on three standard datasets: character recognition (Chars4k~\cite{campos}, accuracy 74.71\%), text recognition (USPS~\cite{hull}, accuracy 94.4\%), and object categorization (WARD~\cite{yang}, accuracy 95.7\%). 
The \bonsai models are already over integers and no port from floating-point is required.
We implement the trained classifiers in \tool for all benchmarks in~\cite{bonsai}
and show the represenative results in Table~\ref{tab:bonsai}.
Out of all benchmarks in~\cite{bonsai}, the dataset WARD requires the largest model.
The column ``depth'' in Table~\ref{tab:bonsai} shows the depth of the tree used by \bonsai. The size of \tool
program grows with the depth of the tree as the straightforward \tool implementation requires a 
loop for each layer of the tree.

To summarize, we have demonstrated that \tool is expressive enough to implement various
state-of-the-art classifiers. We discuss scalability next. 
%dataset | size | gates | and | mul | depth | length | comm | timeL | timeW


\begin{table*}
\begin{tabular}{c|c|c|c |c|c|c|c|c|c | c}
Dataset       & LAN time (s) & WAN time (s) & Comm. (Mb)  & \#And & \#Mul & \#Gates & depth & LOC\\
\hline
Chars4k    &  0.1         & 0.7         & 2            & 18k    & 3k    &  85k     & 1   & 89\\
\hline
USPS       &  0.2         & 0.9         & 4            & 62k    & 2k    &  285k    & 2   & 156\\
\hline
WARD       &  0.3         & 1.1         & 9            & 106k    & 8k    &  506k    & 3   & 283\\
\hline
\end{tabular}

 \caption{Bonsai benchmarks}
 \label{tab:bonsai} 
\end{table*}

\subsection{Pipelining}
\label{sec:pipeeval}
The largest benchmark of \minion~\cite{minionn} is a DNN for CIFAR-10 dataset~\cite{cifar}.
The classifier's task is to categorize colored ($32\times 32$) images into 10 classes. A secure evaluation of this DNN needs more memory than what is available
on our machines. Therefore, we use pipelining and divide the computation into seven stages.
The first step does a convolution with 64 output channels and a ReLU activation.
The next four stages together perform a convolution that involves multiplying a $64\times 576$ matrix with a
$576\times 1024$ matrix. The sixth stage performs a ReLU and a convolution.
The final stage has four convolutions, five ReLUs, and a fully connected layer.
The total number of lines of \tool code for this benchamrk is 336 lines.


The results in Table~\ref{tab:cifar} show the end-to-end results as well as the statistics for the sixth stage (the heaviest stage). 
The number of gates are in millions, hence the suffix `m' in the last three columns.
For this DNN, \minion takes 544 seconds and communicates 9272 Mb. 
Like the results in Table~\ref{tab:nn}, \tool generated implementations are competitive with \minion here as well. Therefore, we believe that (with pipelining) \tool generated generic 2PC implementations can scale to arbitrary sized computations  while maintaining performance
competitive with existing specialized protocols for secure prediction.
In particular, for a large enough DNN, \minion would run out of memory
but a pipelined \tool implementation would still succeed.
\begin{table}
\begin{tabular}{c|c|c|c |c|c|c}
           &  LAN (s) & WAN (s) & Comm. (Mb)  & \#And & \#Mul & \#Gates \\
\hline
Total      &  265.6       & 647.5        & 40683       & 21m    & 61m    &  337m  \\
\hline
Stage 6    &  55.2        & 122.6        & 6744        & 12m    & 10m   &  98m  \\
\hline
\end{tabular}

 \caption{Pipelining results for CIFAR-10. \minion takes 544 seconds and communicates 9272 Mb to evaluate this DNN.}
 \label{tab:cifar} 
\end{table}

%stage | time | comm | gates | mul | add | depth
