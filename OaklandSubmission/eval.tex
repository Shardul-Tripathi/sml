\section{Evaluation}
\label{sec:eval}
We evaluate \tool on a variety of problems that can be thought to fall under the umbrella of \nc{saying it like this so that it does not seem we are evaluating on one problem.} {\it secure prediction}, where one
party (the server) has a machine learning model, and the other party
(the client) has an input. The goal is to compute the output of the
model on client's input, with the guarantee that the server
learns nothing about the input, and the client learns nothing
about the model beyond what is revealed from the output.

To begin, we first implement the benchmarks from Bost et al.~\cite{shafindss}
and \minion~\cite{minionn} (both of which study the same setting),
and show that the performance of the high-level code written in \tool
is comparable to their hand-crafted circuits
(Section~\ref{sec:shallow} and Section~\ref{sec:dnneval}).
%
Next, we demonstrate the generality and programmability aspects of
\tool by implementing state-of-the-art machine learning models from
Tensorflow~\cite{tensorflow} and \bonsai~\cite{bonsai}. Indeed, we
provide the first 2PC implementation of \bonsai.
%
Finally, we
implement a Deep Neural Network (DNN) for CIFAR-10
dataset~\cite{cifar} from \minion~\cite{minionn} to illustrate the use
of partitioning in \tool.

We present the numbers for two network settings, a LAN setting and
a cross-continent WAN setting. The round trip time between the server
and the client machines in the two settings is 1ms and 40ms resp. Each
machine has an Intel(R) Xeon(R) CPU E5-2673 v3 processor running at
2.40GHz with 28 GBs of RAM.
%
%% The  \tool compiler is written in Python and
%% compiles each of our benchmarks in under a second to C++ code that makes calls to the ABY
%% library~\cite{aby}. We use an off-the-shelf solver
%% (SeaHorn~\cite{seahorn}) to check the verification conditions required
%% to prove that the array indices remain in bound. The solver takes less
%% than a minute on our largest benchmark.
Since most of our benchmarks
are related to machine learning, we set up some notation (largely
standard in machine learning) and describe our benchmarks next.

\subsection{Benchmarks description}
%\divya{notation can be moved somewhere... } \\

\noindent(We use $[N]$ to denote $\{0,1,\dotsc, N-1\}$. 
Further, given a vector $x\in\R^d$, we say $\mathsf{argmax}\ x = i$ if
$x_i = \mathsf{max}\ \{x_1,\ldots,x_d\}$.) \nc{Why is notation in brackets?}

We focus on the machine learning models for {\it classification}.
A classifier $C$ uses a trained model to {\it predict} a
label $\ell$ for an input data point $x$. For example, given a
data point which is a tuple of height and weight of an individual,
a classifier might be able to predict a label as ``male'' or ``female''\nc{Is this less sexist?}. The {\it
  accuracy} of a classifier refers to the fraction of data points that
the classifier labels correctly from a given set of test data
points.

\subsubsection*{Standard classifiers}
A {\em binary linear classifier} is one of the simplest classifiers. Here,
the input is a data point $x\in\R^d$,
and the model is a vector $w\in\R^d$. The possible labels are
$\ell\in\{\mathit{true},\mathit{false}\}$ and the classifier is
$C_w\equiv w^Tx>0$, where $\cdot^T$ denotes the matrix transpose
operator. \nc{it might be obvious, but should we say $x > y$ returns 1 if $x>y$ and 0 otherwise?} This classifier requires $d$ multiplications, $d-1$
additions, and a comparison.

A more interesting classifier is {\em Na\"{i}ve Bayes} that predicts labels
from the set $[n]$.
Here, the input data point is a {\it feature}
vector $x=(x_0,x_2,\ldots,x_{d-1})^T$ where each $x_j\in [F]$.
The model has two matrices: a vector $P$ of length $n$ s.t. for each
$i
\in [n]$, $P(i)$ is the log-probability that the output label is $i$.
%$P(i) \equiv \log p(\ell=i)$, where $p(\ell=i)$ is the likelihood
%that the output label is $i$ for $i\in[n]$.
The other matrix $T$ has size $d\times F\times n$ and for each $j\in
[d], k\in [F], i\in [n]$, the entry $T(j)(k)(i)$ is 
%\equiv\log p(x_j|\ell=i)$, i.e.,
the log-probability that  the $j^{th}$ input feature is $k$  conditioned on  the output label being $i$.
The classifier $C_{P,T}(x)$ outputs
\[
{\sf argmax}\ P+\sum_{j=0}^{d-1} T(j)(x_j)
\]
This classifier requires $ndF$ comparisons and additions. Note here  that computing $T(i)(j)(x_j)$ requires a secret look-up in matrix  $T$  based on secret input $x_j$ resulting in $F$ comparisons using multiplexers (see
  Section~\ref{sec:impl}). \nc{Made DG's comment as part of regular text - worth point out}. \cmmt{Is there a +1 etc missing or is this exact?} \nc{By our calculation in Sec 5, it seems that this should be $F/2$ multiplexers}

A {\em decision tree} of size $N$ takes as input an $x\in\R^d$, and the
model consists of a binary tree of depth $d$ with a boolean predicate
assigned to each internal node. The root note is the $0^{th}$ node and
for an internal node $j \in [N]$, the children nodes are $2j+1$ and
$2j+2$. Each internal node $j\in[N]$ at depth $i$ has a predicate
$b_{j}^i\equiv x_i\leq w_{j}$. We start evaluating the tree from the
root and if the predicate at the current node $b_j^i$ is false (resp.,
true) then $x$ is passed to the left (resp., right) child with
predicate $b_{2j+1}^{i+1}$ (resp., $b_{2j+2}^{i+1}$). This process is
repeated till we reach a leaf. The leaves of the tree are labels and
the output label is the leaf visited by this traversal.
Such a (binary) decision tree can be encoded as a polynomial (of
degree linear in depth of the tree) and then the prediction task
reduces to polynomial evaluation. E.g., a binary tree with depth one
and size $N=3$ can be encoded as the polynomial
$(1-b(x))\ell_0 + b(x)\ell_1$ where $b(x)$ is the predicate at the
root, $\ell_0$ is the label of the left leaf and $\ell_1$ is the label
of the right leaf.

\subsubsection*{Deep neural nets}
The next class of classifiers that we benchmark are deep neural nets
or DNNs. A DNN has multiple layers such that each layer computes a
matrix
multiplication followed by an {\it activation} function $f$. The most
common activation functions are square $f(x)=x^2$ and rectifier linear
unit (ReLU) $f(x)=\mathsf{max}(x,0)$.
If $A$ is a matrix then the output of $f(A)$ is also a matrix that is
obtained by applying $f$ to each entry of $A$ pointwise.
Given an input vector $x$, the predicted label of a DNN is
\[
 \mathsf{argmax}\ W_N\cdot f_{N-1}(\ldots f_1(W_1\cdot x)\ldots)
\]
Here, $f_i$'s are the (public) activation functions, the model
consists of matrices $W_i$,  $x \in \R^d$ is the input vector, and the
operator $\cdot$ denotes a matrix multiplication.
Neural nets usually have one or more fully connected layers, each of
which multiplies a matrix with a vector.
Some neural nets have convolution layers and such DNNs are also called
Convolutional Neural Nets or CNNs.
For the purpose of this paper, a convolution can be considered as a
(heavy) matrix-matrix multiplication. The size of matrices manipulated
by a convolution layer grows linearly with {\it window size}
(typically 9 or 25), the number of {\it output channels} (typically
16, 32, or 64), and the size of the matrix input to this layer.
Therefore, fully connected layers are lighter computation-wise compared
to convolution layers. However, the model size of fully connected
layers is larger than those of convolution layers.
In general, DNNs are computationaly heavy but provide
much better accuracies on computer vision tasks than the classifiers
discussed above.

\subsubsection*{State-of-the-art classifiers}
Finally, there are a class of machine learning classifiers that are
much more efficient than
DNNs and provide reasonably good accuracies on standard learning
tasks. \bonsai~\cite{bonsai} is a state-of-the art classifier in this
class and \tool provides the first 2PC protocol for it.
\bonsai takes as input $x \in \R^d$, and its model consists of a
binary tree with $N$ nodes, and a matrix $Z$. Each node $j$ contains
matrices $W_j$ and $V_j$, and a vector $\theta_{j}$. The internal node
$j$  evaluates a predicate $\theta_j^TZx > 0$ to decide whether
to pass $x$ to the left child $2j+1$ or the right child $2j+2$.
The predicted value is

\aseem{I had a hard time parsing this. (1) Should we use $\cdot$ for
  matrix multiplication as we did for DNNs? (2) Precedence of $\circ$
  over scalar multiplication (or is it associative so doesn't matter?)
(3) The argument to $f$ is a scalar (right?), and so is its output,
so then is $\circ$ defined OR is it applying $f$ pointwise to a vector?}
\nc{(1) I agree we should do that to be consistent with before. (2) I assumed $f()$ is computed to give a vector somehow and $I_j(x)W_j^TZx$ also gives a vector somehow and then we do $\circ$; I might be wrong. (3) Hmm.. I thought it was a vector, but I dont know how. (4) Can we specify somewhere the dimensions of these matrices?}
\[
{\sf argmax} \sum_{j = 0}^{N - 1} I_j(x)W_j^TZx\circ f(V_j^TZx) 
\]

Here, $I_j(x)$ is 1 if the $j^{th}$ node is present on the path traversed by $x$
and is zero otherwise. 
The operation $\circ$ is a pointwise dot product, $W_j$'s and $V_j$'s
are matrices. The activation function $f$ is given by $f(y) = y$ if
$-1 < y < 1$ and $\mathrm{sign}(y)$ otherwise. \bonsai can be
seen as a variant of decision trees where the prediction is a function
of the path traversed from root to the leaf and not just the leaf
itself.

In the following, we implement these classifiers in \tool and report
the time taken for making secure predictions. Ideally, the machine
learning classifiers are mathematical expressions over $\R$ that are
usually approximated by floating-point operations. 
As is standard, we port the classifers to integer manipulating
programs by scaling the models and rounding~\cite{minionn}. These
ported classifiers are then implemented in \tool.

\subsection{Standard classifiers}
\label{sec:shallow}
We train the three standard classifiers, linear, Na\"{i}ve Bayes, and
decision trees, on the following data sets from the UCI
machine learning repository~\cite{uci}:
 the Wisconsin Breast Cancer data set, 
Credit Approval data set, Audiology (Standardized) data set, Nursery
data set, and ECG (electrocardiogram) classification data
from~\cite{barni}. \nc{Say here that the training is done on cleartext data?}

\subsubsection*{Linear classifier}The results for linear
classification are in Table~\ref{tab:lc}. The
input and the model are both vectors of length $d$. The columns
``Prev. time'' and ``Prev. comm'' show the time and the total
network communication reported by Bost et al.~\cite{shafindss} for a
network setting with 40ms
round trip time, which is same as our WAN setting. The total execution
time
of \tool generated code in the LAN and the WAN setting is reported
next, followed by the total communication.
We observe that the \tool code performance matches that of Bost
et al., while the programmer effort in \tool is just 20
lines (last column in the table) of high-level code in the \tool
source language, as
opposed to writing low-level circuits. \aseem{They write circuits,
  right?} \nc{Actually, they have a specific protocols for dot product and comparison and also say that their developer effort is minimal as they have to write two function calls to these protocols :).} Our execution time is dominated by the multiplications and
boolean-ands.
The number of boolean-and gates (\#And) remains constant for both the
benchmarks as they reflect the number of gates required for a
single 32-bit integer comparison with zero, while the number of
multiplication gates (\#Mul) is the same as the model size $d$.
%% The
%% last column shows the lines of code of \tool program, which is also
%% independent of the model size.

\subsubsection*{Na\"{i}ve Bayes} The results for Na\"{i}ve Bayes are
in Table~\ref{tab:nb}. As before, $d$ denotes the length of the input
feature vector $x$ and $F$ is the number of possible values of $x_i$
for each $i \in [d]$.
Although, there are no multiplications in these benchmarks, these do
have a significant number of comparisons (due to array look-ups using
secret indices) that
result in many boolean-and gates. %(13k/750k denote 13000/75000). 
As before columns ``Prev. time'' and ``Prev. comm.'' (columns 4 and 5)
show the numbers reported
by Bost et al.~\cite{shafindss}. We observe that \tool generated code
has better performance, despite using a generic 2PC,
as opposed to custom designed protocols developed by Bost et
al. Moreover, they remark that in their setup, generic Yao-based 2PC
did not scale to the smallest of their Na\"{i}ve Bayes classifiers, so
they had to scale down the prediction task, and even then their
implementation was 500x slower. Whereas, we show that by using a
cryptographic-cost aware compiler, we can scale generic 2PC to real
prediction tasks, and get performance competitive to or better than the
specialized protocols.

\subsubsection*{Decision trees}
Our decision tree numbers in Table~\ref{tab:dt} further validate
this claim. We compare with the recent work of Wu et al.~\cite{wu} (columns 3 and 4) who
describe specialized protocols for trees and forests (that perform
better than~\cite{shafindss}).
Similar to Bost et al.~\cite{shafindss}, they also remarks that these
computations are infeasible with Yao-based 2PC.
Again, we observe that
the performance of \tool generated generic 2PC
code is competitive with specialized protocols. Therefore, we believe
that \tool generated code can be a better baseline than Yao-based 2PC
when comparing against generic 2PC-based implementations for secure
prediction tasks. \aseem{I don't understand this last sentence. Why is
  ours a better baseline?} \nc{This last line is more a comment for a ``conclusion'' and is a bit out of place here. I guess we want to say that till recently if generic 2PC using Yao was not feasible, then people had to come up with specialized protocols and didnt have any benchmark or had a poor benchmark (of Yao) to compare their 2PC with. Moving forward, our language generated 2PC can be the baseline benchmark for all 2PC protocols.}

\begin{table*}
\begin{tabular}{c|c|c|c|c |c|c|c|c|c|c}
Dataset & $d$  & Prev. time (s) & Prev. Comm (kb) & LAN time (s) & WAN time (s) & Comm. (kb)  & \#And & \#Mul & \#Gates & LOC\\
\hline
Breast cancer & 30 & 0.3 & 36 & 0.1 & 0.3 & 25 & 95 & 30 & 727 & 20\\
\hline
Credit & 47 & 0.3 & 41 & 0.1 & 0.3 & 36 & 95 & 47 & 795 & 20\\
\hline
\end{tabular}

 \caption{Linear classification results. We compare our results (columns 5, 6, 7) with~\cite{shafindss} (columns 3 and 4)}
 \label{tab:lc} 
\end{table*}

\begin{table*}
\begin{tabular}{c|c|c|c|c|c |c|c|c|c|c|c}
Dataset & $d$ & $F$ & Prev. time (s) & Prev. Comm (Mb) & LAN time (s) & WAN time (s) & Comm. (Mb)  & \#And & \#Mul & \#Gates  & LOC\\
\hline
Nursery & 5 & 9 & 1.5 & 0.2 & 0.1 & 0.4 & 0.6 & 13k & 0 & 73k & 50\\
\hline
Audiology & 24 & 70 & 3.9 & 2.0 & 1.5 & 2.9 & 37 & 750k & 0 & 4219k & 50\\
\hline
\end{tabular}

 \caption{Na\"{i}ve Bayes results. We compare our results (columns 6 , 7, 8) with~\cite{shafindss} (columns 4 and 5)}
 \label{tab:nb} 
\end{table*}

\begin{table*}
\begin{tabular}{c|c|c|c |c|c|c|c|c|c|c}
Dataset  & $N$ & Prev. time (s) & Prev. Comm (kb) & LAN time (s) & WAN time (s) & Comm. (kb)  & \#And & \#Mul & \#Gates & LOC\\
\hline
Nursery & 4 & 0.3 & 102 & 0.1 & 0.3 & 32 & 504 & 3 & 3324 & 20\\
\hline
ECG &  6 & 0.4 & 102 & 0.1 & 0.4 & 49 & 756 & 5 & 5002 & 20\\
\hline
\end{tabular}

 \caption{Decision tree benchmarks. We compare our results (columns 5 , 6, 7) with~\cite{wu} (columns 3 and 4)}
 \label{tab:dt} 
\end{table*}


%dataset  | size | gates | and | mul | depth | length

%dataset | time | comm | timeL | timeW | comm
\subsection{Deep neural nets}
\label{sec:dnneval}
We evaluate \tool on the DNNs described in SecureML~\cite{secureml},
Cryptonets~\cite{cryptonets}, and CNN from \minion~\cite{minionn}. For
comparison, we consider their implementations from
\minion~\cite{minionn}, which outperforms their previous
implementations. Table~\ref{tab:nn} shows the
results\footnote{\minion does not report the network round-trip time.}.
We note that for each of these DNNs, \minion provides a
specialized protocol, while \tool uses a generic 2PC protocol
(auto) generated from high-level code.

The first benchmark  is the DNN described in
SecureML~\cite{secureml}.
It has three fully connected layers with square as the activation function.
This DNN is light-weight as it requires only arithmetic computations
(additions and multiplications) 
except for the $\mathsf{argmax}$ at the end.
Next, we implement the DNN described in Cryptonets~\cite{cryptonets}
in \tool.
This DNN also uses square as the activation function and has one
convolution (with 5 output channels) and one fully connected layer. 
Finally, we implement CNN from \minion, that has
two convolutions (with 16 output channels each) and two fully
connected layers.  In contrast to the previous two DNNs, it uses
ReLU for activation and
has significantly higher number of boolean-and gates.
Recall, that square activation can be implemented entirely using
arithmetic gates but ReLU requires boolean-and gates.  For a complete
description of these benchmarks and their accuracies, we refer the
reader to the original references. 

In Table~\ref{tab:nn}, columns 2 and 3 show the time taken by \minion.
The column ``Model size'' is the number of 32 bit integers in the
trained model.
We observe that our performance is competitive with specialized
\minion protocols, for both LAN and the WAN settings. Further, lines
of \tool source code required is still small. 
We note that \minion and \tool implementations are both based on ABY.
These results are surprising as \minion has a much more efficient
preprocessing phase and has SIMD 
(single instruction multiple data) capabilities that allow it to perform
matrix operations efficiently. In contrast, \tool focuses on generic
2PC, and such optimizations are part of our future work.
\minion also reports performance results on a bigger DNN with 7
convolution layers. In \tool, this benchmark requires partitioning and
we discuss it in Section~\ref{sec:pipeeval}.
%net | minionT | minionC | timeL | timeW | comm | gates | and | mul | length 

\begin{table*}
\begin{tabular}{c|c|c|c |c|c|c|c|c|c|c}
DNN  & Prev. time (s) & Prev. Comm (Mb) & LAN time (s) & WAN time (s) & Comm. (Mb)  & \#And & \#Mul & \#Gates & Model Size & LOC\\
\hline
SecureML   &  1.1 & 15.8 & 0.7 & 1.7  & 76   &  2k   & 119k & 366k   & 119k & 78\\
\hline
Cryptonets &  1.3 & 47.6 & 0.6 & 1.6  & 70    & 2k    & 108k & 316k & 86k & 88\\
\hline
CNN        &  9.4 & 657.5& 5.1 & 11.6 & 501  & 1640k & 667k & 9480k & 35k & 154\\
\hline
\end{tabular}

 \caption{DNN benchmarks. We compare our results (columns 4 , 5, 6) with~\cite{minionn} (columns 2 and 3)}
 \label{tab:nn} 
\end{table*}


\subsection{State-of-the-art classifiers}
Tensorflow~\cite{tensorflow} is a standard machine learning toolkit.
Its introductory tutorial describes two prediction models for handwritten digit recognition
using the MNIST dataset~\cite{mnist}.
Each image in this dataset is a greyscale $28\times 28$ image of
digits 0 to 9.
The first model that the tutorial describes is a softmax regression
that provides an accuracy of 92\%. The classifier evaluates
\[
 \mathsf{argmax}\ Wx+b
\]
Here, $i\in [10]$, $x$ is a 784 length vector obtained from the input image,
each $W$ is a $10\times 784$ matrix, and each $b$ is a $784$ length
vector. \aseem{Questions on this expression: (1) where is $i$?
  (2) Should $b$ be 10 length vector? (3) Again, should be use $\cdot$
  for matrix multiplication
as we did earlier to be consistent?}
We implement this classifer in \tool and present the results in the
first row of Table~\ref{tab:tf}.
Since, we are not aware of any other tools that has used this
model as a benchmark, we only report numbers for \tool.


The next classifer in the Tensorflow tutorial is a convolution neural net with two convolutions
(with 32 output channels) and two fully connected layers with ReLU as the activation function.
This DNN is both bigger and more accurate than the DNNs presented in the previous section.
In particular, it has an accuracy of 99.2\%\footnote{The accuracy of
  the integer model is 0.04\% higher than the floating-point
  model.}. \aseem{Is this footnote necessary? As a reader, I won't
  miss it if it was not there.}
We observe that this DNN can take a minute per prediction in the WAN
setting and is the largest benchmark that we have evaluated without
partitioning.


% benchmark | size | gates | and | mul | depth  | length | comm | timeL | timeW

\begin{table*}
\begin{tabular}{c|c|c|c |c|c|c|c|c|c | c}
Classifier       & LAN time (s) & WAN time (s) & Comm. (Mb)  & \#And & \#Mul & \#Gates & Model size & LOC\\
\hline
Regression &  0.1         & 0.7         & 5            & 2k    & 8k    &  35k    & 8k   & 38\\
\hline
CNN        &  30.5        & 60.3        & 2955         & 6082k & 4163k &  42104k & 3226k& 172\\
\hline
\end{tabular}

 \caption{Tensorflow tutorial benchmarks}
 \label{tab:tf} 
\end{table*}


We next present \bonsai~\cite{bonsai} results on three
standard datasets: character recognition (Chars4k~\cite{campos},
accuracy 74.71\%), text recognition (USPS~\cite{hull}, accuracy
94.4\%), and object categorization (WARD~\cite{yang}, accuracy
95.7\%). 
The \bonsai models are already over integers and no porting from
floating-point is required.
We implement the trained classifiers in \tool for all the benchmarks
from~\cite{bonsai},
and show the represenative results in Table~\ref{tab:bonsai}.
Out of all the benchmarks from~\cite{bonsai}, the dataset WARD
requires the largest model. 
The column ``depth''  shows the depth of the
tree used by \bonsai. The size of \tool 
program grows with the depth of the tree, as the straightforward \tool
implementation requires a loop for each layer of the
tree. \aseem{Should we remark here that once we have functions in the
  language, LOC would come down, perhaps in the footnote? Else a PL
  reviewer might scoff at it :).}

To summarize, by providing first 2PC implementations of
state-of-the-art classifiers, we have demonstrated the expressiveness
of \tool. We discuss scalability next.
%dataset | size | gates | and | mul | depth | length | comm | timeL | timeW


\begin{table*}
\begin{tabular}{c|c|c|c |c|c|c|c|c|c | c}
Dataset       & LAN time (s) & WAN time (s) & Comm. (Mb)  & \#And & \#Mul & \#Gates & depth & LOC\\
\hline
Chars4k    &  0.1         & 0.7         & 2            & 18k    & 3k    &  85k     & 1   & 89\\
\hline
USPS       &  0.2         & 0.9         & 4            & 62k    & 2k    &  285k    & 2   & 156\\
\hline
WARD       &  0.3         & 1.1         & 9            & 106k    & 8k    &  506k    & 3   & 283\\
\hline
\end{tabular}

 \caption{Bonsai benchmarks}
 \label{tab:bonsai} 
\end{table*}

\subsection{Partitioning}
\label{sec:pipeeval}
The largest benchmark of \minion~\cite{minionn} is a DNN for CIFAR-10
dataset~\cite{cifar}. 
The classifier's task is to categorize colored ($32\times 32$) images
into 10 classes. A secure evaluation of this DNN needs more memory
than what is available
on our machines \aseem{Is the limitation machine RAM or ABY
  limit?}. Therefore, we use partitioning and divide the
computation into seven stages. 
The first step does a convolution with 64 output channels and a ReLU
activation. The next four stages together perform a convolution that
involves multiplying a $64\times 576$ matrix with a 
$576\times 1024$ matrix. The sixth stage performs a ReLU and a
convolution. The final stage has four convolutions, five ReLUs, and a
fully connected layer. The total number of lines of \tool code for
this benchamrk is 336 lines. \aseem{Would it be better if we had
  functions, for loop expressions etc.? If so, we should add a
  footnote.}


Table~\ref{tab:cifar} shows the end-to-end numbers as
well as the numbers for the sixth (the heaviest) stage. 
The number of gates are in millions, hence the suffix `m' in the last
three columns.
For this DNN, the specialized \minion protocol takes 544 seconds and
communicates 9272 Mb.
As with Table~\ref{tab:nn}, \tool generated generic 2PC protocol is
competitive with \minion here as well. Therefore,
we believe that with partitioning, \tool can scale to arbitrary sized
computations while maintaining performance
competitive with existing specialized protocols. In particular, for a
large enough DNN, \minion would run
out of memory but a pipelined \tool implementation would still
succeed. \aseem{Last sentence is speculative? Should we say \minion
  would or could or might?}

\begin{table}
\begin{tabular}{c|c|c|c |c|c|c}
           &  LAN (s) & WAN (s) & Comm. (Mb)  & \#And & \#Mul & \#Gates \\
\hline
Total      &  265.6       & 647.5        & 40683       & 21m    & 61m    &  337m  \\
\hline
Stage 6    &  55.2        & 122.6        & 6744        & 12m    & 10m   &  98m  \\
\hline
\end{tabular}

 \caption{Partitioning results for CIFAR-10. \minion takes 544 seconds and communicates 9272 Mb to evaluate this DNN.}
 \label{tab:cifar} 
\end{table}

%stage | time | comm | gates | mul | add | depth
