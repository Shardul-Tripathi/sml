\usepackage{epsfig,endnotes}
\usepackage{varwidth}
\usepackage{algorithm}
\usepackage{algpseudocode}
\usepackage{balance}
\usepackage{xcolor}
\usepackage{nicefrac}
\usepackage{amsmath}
\usepackage{braket}
\usepackage{bm}
\usepackage{mathtools}
\usepackage{multirow}
\usepackage{bigdelim}
\usepackage{mathtools}
\usepackage{amssymb}
\usepackage{indentfirst}
\usepackage{booktabs}
\usepackage{enumitem}
\usepackage{boxedminipage}
\usepackage{float}
\usepackage{graphicx}
\usepackage{caption}
\usepackage{cleveref}
\usepackage{subcaption}
\usepackage[normalem]{ulem}
\usepackage{xpatch}
%\usepackage{MnSymbol}
\usepackage{xspace}
\usepackage{listings}
\usepackage{mathpartir}

\usepackage[pdftex,bookmarks=true,pdfstartview=FitH,colorlinks,linkcolor=blue,filecolor=blue,citecolor=blue,urlcolor=blue,pagebackref=true]{hyperref}
    \urlstyle{sf}

\usepackage{stmaryrd}

\newlength{\saveparindent}
\setlength{\saveparindent}{\parindent}
\newlength{\saveparskip}
\setlength{\saveparskip}{\parskip}
 
 
\newenvironment{tiret}{%
\begin{list}{\hspace{2pt}\rule[0.5ex]{6pt}{1pt}\hfill}{\labelwidth=15pt%
\labelsep=5pt \leftmargin=20pt \topsep=3pt%
\setlength{\listparindent}{\saveparindent}%
\setlength{\parsep}{\saveparskip}%
\setlength{\itemsep}{0pt} }}{\end{list}}
 
\newenvironment{enum}{%
\begin{list}{{\rm (\arabic{ctr})}\hfill}{\usecounter{ctr}\labelwidth=17pt%
\labelsep=6pt \leftmargin=23pt \topsep=5pt%
\setlength{\listparindent}{\saveparindent}%
\setlength{\parsep}{\saveparskip}%
\setlength{\itemsep}{3pt} }}{\end{list}}
 
\newenvironment{newenum}{%
\begin{list}{{\rm \arabic{ctr}.}\hfill}{\usecounter{ctr}\labelwidth=17pt%
\labelsep=6pt \leftmargin=23pt \topsep=.5pt%
\setlength{\listparindent}{\saveparindent}%
\setlength{\parsep}{\saveparskip}%
\setlength{\itemsep}{5pt} }}{\end{list}}

\newcommand{\tool}{{\textsc{EzPC}}\xspace}
\newcommand{\minion}{{\textsc{MiniONN}}\xspace}
\newcommand{\bonsai}{{\textsc{Bonsai}}\xspace}
\newcommand{\R}{{\mathbb{R}}\xspace}
\newcommand{\mpc}{{2PC}\xspace}

\newtheorem{problem}{Problem}
\newtheorem{theorem}{Theorem}
\newtheorem{conjecture}[theorem]{Conjecture}
\newtheorem{definition}[theorem]{Definition}
\newtheorem{lemma}[theorem]{Lemma}
\newtheorem{proposition}[theorem]{Proposition}
\newtheorem{corollary}[theorem]{Corollary}
\newtheorem{claim}[theorem]{Claim}
\newtheorem{fact}[theorem]{Fact}
\newtheorem{remk}[theorem]{Remark}
\newtheorem{apdxlemma}{Lemma}


\newcommand{\namedref}[2]{\hyperref[#2]{#1~\ref*{#2}}\xspace}
\newcommand{\lemmaref}[1]{\namedref{Lemma}{lem:#1}}
\newcommand{\propref}[1]{\namedref{Proposition}{prop:#1}}
\newcommand{\theoremref}[1]{\namedref{Theorem}{theorem:#1}}
\newcommand{\claimref}[1]{\namedref{Claim}{clm:#1}}
\newcommand{\corolref}[1]{\namedref{Corollary}{corol:#1}}
\newcommand{\figureref}[1]{\namedref{Figure}{fig:#1}}
\newcommand{\tableref}[1]{\namedref{Table}{tbl:#1}}
\newcommand{\equationref}[1]{\namedref{Equation}{eq:#1}}
\newcommand{\defref}[1]{\namedref{Definition}{def:#1}}
\newcommand{\observationref}[1]{\namedref{Observation}{obs:#1}}
\newcommand{\procedureref}[1]{\namedref{Procedure}{proc:#1}}
\newcommand{\importedtheoremref}[1]{\namedref{Imported Theorem}{impthm:#1}}
\newcommand{\informaltheoremref}[1]{\namedref{Informal Theorem}{infthm:#1}}

\newcommand{\sectionref}[1]{\namedref{Section}{sec:#1}}
\newcommand{\appendixref}[1]{\namedref{Appendix}{app:#1}}
\newcommand{\propertyref}[1]{\namedref{Property}{prop:#1}}

\newcommand{\algoref}[1]{\namedref{Algorithm}{algo:#1}}



\newcommand{\TODO}[1]{{{\color{red} TODO: #1}}}
\newcommand{\divya}[1]{{{\color{blue} dg: #1}}}
\newcommand{\cmmt}[1]{{{\color{red} check: #1}}}
\newcommand{\nc}[1]{{{\color{purple} nc: #1}}}
\newcommand{\rs}[1]{{{\color{magenta} rs: #1}}}

\definecolor{mypink}{rgb}{1,0.2,0.4}
\newcommand{\aseem}[1]{{{\color{mypink} Aseem: #1}}}

\newcommand{\adv}{\mathcal{A}}
\newcommand{\env}{\mathcal{Z}}
\newcommand{\prot}{\Pi}
\newcommand{\real}{\mathsf{REAL}}
\newcommand{\ideal}{\mathsf{IDEAL}}
\newcommand{\simu}{\mathcal{S}}
\newcommand{\F}{\mathcal{F}}
\newcommand{\secparam}{\kappa}


%%circuits syntax
\newcommand{\gate}{\mathtt{g}}
\newcommand{\wire}{w}
\newcommand{\bval}{\tilde{r}}
\newcommand{\val}{\tilde{v}}
\newcommand{\crct}{\chi}

\lstset{ % 
    language=C,
    backgroundcolor=\color{white},   
    basicstyle=\footnotesize\ttfamily\bfseries,
    breakatwhitespace=false,
    breaklines=false,
    belowskip=-0.3cm,
    captionpos=b,                    
    commentstyle=\color{red},
    deletekeywords={...}, 
    escapeinside={\%*}{*)}, 
    extendedchars=true, 
    %% frame=single,
    keepspaces=true,
    keywordstyle=\color{blue},
    keywordstyle=[2]\color{blue},
    otherkeywords={*,...,in,uint,input1,input2,output},
    keywords=[2]{private},
    numbers=left,
    numbersep=5pt, 
    numberstyle=\tiny\color{gray}\bfseries, 
    rulecolor=\color{black},
    showspaces=false,
    showstringspaces=false, 
    showtabs=false, 
    stepnumber=2, 
    stringstyle=\color{mymauve},
    tabsize=2, 
    title=\lstname
}

\let\ls\lstinline
