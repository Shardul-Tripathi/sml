


\documentclass[conference,compsoc]{IEEEtran}

\usepackage{epsfig,endnotes}
\usepackage{varwidth}
\usepackage{algorithm}
\usepackage{algpseudocode}
\usepackage{balance}
\usepackage{color}
\usepackage{nicefrac}
\usepackage{amsmath}
\usepackage{braket}
\usepackage{bm}
\usepackage{mathtools}
\usepackage{multirow}
\usepackage{bigdelim}
\usepackage{mathtools}
\usepackage{amssymb}
\usepackage{indentfirst}
\usepackage{booktabs}
\usepackage{enumitem}
\usepackage{boxedminipage}
\usepackage{float}
\usepackage{graphicx}
\usepackage{caption}
\usepackage{cleveref}
\usepackage{subcaption}
\usepackage[normalem]{ulem}
\usepackage{xpatch}
\usepackage{stmaryrd}


\makeatletter
\xpatchcmd{\algorithmic}{\itemsep\z@}{\itemsep=0.3ex plus2pt}{}{}
\makeatother

\renewcommand{\algorithmicrequire}{\textbf{Input:}}
\renewcommand{\algorithmicensure}{\textbf{Output:}}
\newcommand{\crefrangeconjunction}{ to~}
\newcommand{\minus}{\scalebox{0.75}[1.0]{$-$}}
\newcommand{\sameer}[1]{{\leavevmode\color{red}{#1}}}
\setlength{\textfloatsep}{0.3cm}
\newcommand{\ourprotocol}{Tuffy}

\newcommand{\bits}{\{0,1\}}
\newcommand{\Ex}{\mathbb{E}}
\newcommand{\To}{\rightarrow}
\newcommand{\R}{\mathbb{R}}
\newcommand{\N}{\mathbb{N}}
\newcommand{\Z}{\mathbb{Z}}
\newcommand{\F}{\mathbb{F}}
\newcommand{\GF}{\mathsf{GF}}

\usepackage{MnSymbol}
\newcommand{\maxpr}{\text{\rm max-pr}}
\newcommand{\cclass}[1]{\mathbf{#1}}
\renewcommand{\P}{\cclass{P}}
\newtheorem{problem}{Problem}
\newtheorem{theorem}{Theorem}
\newtheorem{conjecture}[theorem]{Conjecture}
\newtheorem{definition}[theorem]{Definition}
\newtheorem{lemma}[theorem]{Lemma}
\newtheorem{proposition}[theorem]{Proposition}
\newtheorem{corollary}[theorem]{Corollary}
\newtheorem{claim}[theorem]{Claim}
\newtheorem{fact}[theorem]{Fact}
\newtheorem{remk}[theorem]{Remark}
\newtheorem{apdxlemma}{Lemma}

\newcommand{\etal}{{\it et~al.\ }}
\newcommand{\ie} {{\it i.e.,\ }}
\newcommand{\eg} {{\it e.g.,\ }}
\newcommand{\cf}{{\it cf.,\ }}
\newcommand{\suchthat}{{\;\; : \;\;}}
\newcommand{\pr}[1]{\Pr\left[#1\right]}

\newcommand{\class}[1]{\mathbf{#1}}
\newcommand{\SZK}{\class{SZK}}
\newcommand{\BPP}{\class{BPP}}
\newcommand{\NP}{\class{NP}}
\renewcommand{\P}{\class{P}}
\newcommand{\negl}{{\mathrm{neg}}}

\newcommand{\Enc}{E}
\newcommand{\Dec}{D}
\newcommand{\Gen}{G}

\newcommand{\PK}{\mathit{PK}}
\newcommand{\SK}{\mathit{SK}}
\newcommand{\pk}{\mathit{pk}}
\newcommand{\sk}{\mathit{sk}}
\newcommand{\MD}[1]{\mathrm{MD{#1}}}
\newcommand{\SHA}{\mbox{SHA-1}}




\begin{document}
\title{SecureDNN: Efficient Analytics over Encrypted Data}


% author names and affiliations
% use a multiple column layout for up to three different
% affiliations
% \author{\IEEEauthorblockN{Michael Shell}
% \IEEEauthorblockA{School of Electrical and\\Computer Engineering\\
% Georgia Institute of Technology\\
% Atlanta, Georgia 30332--0250\\
% Email: http://www.michaelshell.org/contact.html}
% \and
% \IEEEauthorblockN{Homer Simpson}
% \IEEEauthorblockA{Twentieth Century Fox\\
% Springfield, USA\\
% Email: homer@thesimpsons.com}
% \and
% \IEEEauthorblockN{James Kirk\\ and Montgomery Scott}
% \IEEEauthorblockA{Starfleet Academy\\
% San Francisco, California 96678-2391\\
% Telephone: (800) 555--1212\\
% Fax: (888) 555--1212}}



\maketitle
% \begin{abstract}
% The abstract goes here.
% \end{abstract}
\IEEEpeerreviewmaketitle


\graphicspath{{./Images/}}
\DeclareGraphicsExtensions{.pdf,.jpg,.png}

\input{./Sections/CompleteProtocol.tex}
\newpage
\clearpage
\input{./Sections/ReLU.tex}
\input{./Sections/Independent.tex}
\input{./Sections/Introduction.tex}


\newpage
\bibliographystyle{abbrv}
\bibliography{./bib}
\balance

\end{document}


