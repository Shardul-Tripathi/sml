  
  
  %% \TODO{While taking a pass on intro, check citation correctness or if we are missing any citation.}

\section{Introduction}
\label{sec:intro}

%% \aseem{The intuition of this paragraph is conveyed in rest of the
%% intro already?}

 Today it is hard for the developers to program secure applications
 using cryptographic techniques. Typical developers lack a deep
 understanding of cryptographic protocols, and therefore cannot be
 expected to use them correctly and efficiently on their own.
 Ideally, a developer would declare the functionality in a general
 purpose, high-level programming language and a
 tool, e.g. a compiler, would generate an efficient protocol that
 implements the functionality securely, while hiding the cryptography
 behind-the-scenes. 
 
 This paper presents such a framework for Secure
 Two-party Computation (\mpc),
%
%Most cryptographic tools rely on circuit as abstractions for any general computation but real-life developers write programs. It is critical to address this fundamental  representation gap to make cryptography accessible to non-crypto specialist developers. In this work, we address this gap for the case of secure multi-party computation.
%
%Most cryptographic tools are very hard to use non-crypto specialist developers. The task becomes even more tedious and error-prone for the case of secure multiparty computation since different protocol implementations require the function to be expressed in different representations such as Boolean or Arithmetic circuits. To make cryptography accessible to non-crypto specialist developers, we need to design systems that do crypto automatically and satisfy the properties of ... 
%
%Secure Two-party Computation (\mpc)~(\cite{yao,gmw}) is
a powerful
cryptographic technique that allows two mutually distrusting parties
to compute a publicly known joint function of their secret inputs in a way that both
the parties learn nothing about the inputs of each other beyond what
is revealed by their (possibly different) outputs. For example, \mpc
can be used for {\em secure
prediction}~(\cite{shafindss,wu,barni,minionn,secureml}),
where one party (the server) holds a proprietary classifier to predict
a label (e.g. a disease, genomics, or spam detection), and the other
party (the client) holds a private input that it wants to run the
classifier on. Using \mpc guarantees that the server learns nothing
about the client's input or output, and that the client learns nothing
about the classifier, beyond what is revelaed by the output
label.

%Since its introduction in 1980s, there has been a long
%sequence of work that has transformed 2PC from a mere theoretical tool to
%something practically efficient and usable.
%Our goal is to build a framework that enables programmers, who are not
%experts in cryptography, to easily program efficient and scalable \mpc
%protocols. 
To understand the
state-of-the-art, let us consider an example underlying many
secure prediction algorithms. Say Alice wants to write
a \mpc protocol to securely
compute $w^Tx >b$. Here $w$ (a vector) and $b$ (a scalar) constitute
the server classifier, and $x$ is the client's input vector. Further,
$\cdot^{T}$ is the matrix transpose operator, and $w^Tx$ denotes the
inner product of  $w^{T}$ and $x$. Alice has following options.

%\begin{tiret}
%\item

She can program the computation in one of the several programmer
friendly, domain-specific languages (such
as Fairplay~\cite{fairplay}, Wysteria~\cite{wysteria},
ObliVM~\cite{oblivm}, CBMC-GC~\cite{cbmcgc}, SMCL~\cite{smcl},
Sharemind~\cite{sharemind}, \cite{lambdaps} etc.) that would
automatically compile it to a \mpc protocol. However, all of these
frameworks use cryptographic backends that take as input
the computation expressed as either a boolean
circuit~(\cite{yao,gmw}) or an arithmetic circuit~(\cite{sss,viff,gentry}). 
%
The efficiency of the generated \mpc protocol is thus bounded by the
efficiency of representing the computation in \emph{one} of
these representations. For instance, multiplication of two
$\ell$-bit integers can either be expressed as a boolean circuit of
size $O(\ell^2)$, or as an arithmetic circuit with 1 multiplication
gate.
%
For better efficiency, Alice would ideally like to compute
$w^Tx$ using an arithmetic circuit, and the comparison with $b$ using
a boolean circuit.
%
Unfortunately, none of the above frameworks support combinations of
arithmetic and boolean circuits, and using different tools for
different parts of the computation is cumbersome and error-prone.

%% In fact, most interesting functions require a mix of arithmetic and
%% boolean computations for efficiency.

%\item

Alternatively, Alice can use a tool such as ABY (Demmlet et
al.~\cite{aby}) that allows
the computation to be expressed as a combination of arithmetic and
boolean circuits. However, here, the programming interface is quite
low-level -- the programmer is required to first manually split the
computation into arithmetic and boolean components, and then write the
circuits for all the components manually, including the appropriate
inter-conversion gates between them. Clearly, writing correct
and efficient protocols in such a framework is beyond an average
programmer who does not understand the various trade-offs between
arithmetic and boolean circuits, and even for an expert cryptographer,
writing large computations in such a framework can be tedious and
error-prone (a sentiment echoed by Demmler et al.~\cite{aby} themselves).

%\item
A third option for Alice is to earn a PhD in cryptography, and design
and implement specialized, efficient \mpc protocols (similar
to~\cite{shafindss,wu,minionn,valeriaMatrix}) for her tasks.
%\end{tiret}
%In our work, we achieve the best of all the above options. 




%Unfortunately, implementing functionalities using \mpc protocols requires thorough understanding of cryptography. To allow for widespread use of \mpc, it is critical that \mpc protocols are programmable by non-cryptographic experts. To cater to this, there have been several efforts developing domain-specific languages that are programmer friendly and compile to a SMC protocol. Most notable of these domain specific languages and systems include \cite{...}. All of these works build on a cryptographic back-end that is either entirely boolean \cite{yao,gmw} or arithmetic \cite{homo}. In fact, most of these works use boolean circuits based scheme for completeness\footnote{Note that comparisons cannot be expressed in arithmetic circuits}. However, most interesting functions require a mix of arithmetic and boolean computations.  Examples include ......... As one can expect, compiling these programs to boolean circuits is one of the biggest efficiency bottlenecks. \divya{Our work addresses this performance bottleneck.}

%To address this issue \divya{on the cryptographic side}, recently Demmler et al. \cite{aby} gave a cryptographic protocol for SMC where the parties can mix arithmetic and boolean computations and claimed potentially great performance benefits. However, using their system requires the programmer to be aware of trade-offs of arithmetic and boolean cryptographic schemes. In their framework, the programmer is required to write circuits consisting of a mix of arithmetic and boolean gates along with appropriate conversion gates. In short, as is also mentioned by the authors themselves, the current system is not suitable to be used by non-specialist programmers.


%%%%%%%Recent works \cite{aby,secureml,minion} have shown that great performance benefit can be obtained by mixing arithmetic and boolean computations in the back-end. Certain computations such as integer multiplication are efficient when implemented using arithmetic SMC \cite{gmw} whereas max(x,0) needs to be computed using a boolean back-end \cite{yao}. More details later.. ABY \cite{aby} requires programmer to be aware of arithmetic and boolean trade-offs and  write high-level circuits consisting of both arithmetic and boolean gates and share-conversion gates. Other works such as \cite{secureml,minion} build on ideas from \cite{aby} and develop tailor made algorithms for neural network training and prediction \cite{ml} and claim huge improvements over only boolean implementations. As is already mentioned in these works, their systems are mere proofs-of-concept and far from being implementable.



%\divya{old para...} In this work, we develop and implement our framework \tool\footnote{\tool, read as ``easy peasy'', stands for Easy 2 Party Computation} for secure
%computation that achieves generality, performance and programmer
%productivity. The programmer writes a high level \divya{C-like}
%program (instead of a circuit) to describe the function to be
%computed. Our compiler automatically generates an \mpc protocol using a
%mix of Arithmetic and Boolean compute. Our compiler is general and can
%work with any secure implementation of a mix of arithmetic and boolean
%computations. In our work, we focus on semi-honest secure two-party
%computation. We build on ABY \cite{aby} that provides the suitable
%cryptographic back-end in this setting. We provide a formal type
%system and prove correctness and security of our compiler. We evaluate
%our system on various benchmarks such as logistic regression and
%convolutional neural network (CNN) for MNIST data \cite{minionn},
%naive bayes, decision trees from \cite{shafindss}. Evaluations show
%that protocols generated by our compiler match
%the performance or outperform hand-crafted protocols in most cases (see
%\sectionref{eval}). Below, we will  give an overview of \tool and
%describe our contributions in more detail.

%The programmer is oblivious of the cryptographic back-end being used. Our compiler automatically compiles it to a circuit framework consisting to both arithmetic and boolean gates as well share-conversion gates wherever required. We use ABY as the cryptographic backend. \divya{Say something about what kind of backend we want. provides \mpc for a mix of arithmetic and boolean circuit with appropriate secure conversion between two types and ABY provides such a framework for semi-honest 2pc.} Our work is compatible with malicious, multiparty as well... etc We give a type system, correctness ..... We evaluate our framework on ... and show performance comparable and even better than tailor made protocols. It is simple to program in our framework and our work provides a meaningful baseline for future works designing tailor made protocols for specific functionalities.

%\subsubsection*{\tool}
%\label{sec:contrib}
%Write high level overview of programming language and tool.. \divya{have some preamble}

This paper presents \tool\footnote{Read as ``easy peasy'',
stands for Easy 2 Party Computation.}, the first
``cryptographic-cost aware'' framework that generates efficient and
scalable \mpc protocols from high-level programs devoid of any
cryptographic details. The generated protocols use combinations of
arithmetic and boolean circuits and have performance comparable to
or better than the custom, specialized
protocols from previous works~\cite{shafindss,wu,minionn,secureml,cryptonets,valeriaMatrix}.
In fact, these papers (and others) cite the inefficiency of generic \mpc as the major motivation
behind the design of specialized protocols. Using \tool, we empirically  demonstrate
the surprising fact that generic \mpc implementations are much more efficient than what they were believed to be. 
 %(up to a factor of 19x in our
%experiments, depending on the functionality) \nc{I added this but I
%  guess we are saying it many times, but might be nice to highlight
%  perf early no?} \aseem{My rationale was to save the numbers for
%  Evaluation summary and remain high-level here.} 
Below we describe
the salient features of \tool.

\begin{figure}
\begin{lstlisting}[language=C]
uint w[30] = input1(); uint b = input1();
uint x[30] = input2();
uint acc = 0;
for i in [0:30] { acc =  acc + (w[i] * x[i]); }
output2((acc > b) ? 1 : 0) //only to party 2
\end{lstlisting}
\caption{\tool code for $w^Tx >b$}
\label{fig:ex-sml}
\end{figure}

\subsubsection*{Ease of programming} \tool source programs are ideal
functionalities that describe ``what'' computation needs to be done,
rather than ``how'' to do it. In particular, the programmer writes the
high-level computation without thinking about the underlying
cryptographic details. For example, Figure~\ref{fig:ex-sml}
shows an \tool source
program for $w^Tx >b$.
%
%\vspace{0.2cm}
%
%\vspace{0.2cm}
The program is quite similar to what a programmer might
write in C++ or Java. The simplicity of the
language comes with its usual benefits: it is easily accessible to the
developers, there are fewer avenues for making mistakes, developers
don't bear the burden of getting cryptographic details right, 
code optimizations can be left to the compiler,
and it
is easy to maintain and modify the programs. Needless to say,
frameworks that expose low-level circuit APIs to the programmer do not
enjoy these benefits.

\subsubsection*{Cryptogaphic-cost aware compiler} The \tool compiler
compiles a source program to a hybrid computation consisting
of \emph{public} and \emph{secret} parts. In the example above, for
instance, \tool compiler realizes that the array index~\ls{i} is
public, and generates non-cryptographic code for the array accesses.
Further, within the secret parts, \tool compiler is
aware
of the cryptographic costs of arithmetic and boolean representations
of the source language operators. Based on these costs, the compiler
automatically picks arithmetic or boolean
representations for different sub-parts, and generates the
corresponding circuits along with the required inter-conversion
gates. The outcome is an effcient \mpc protocol combining arithmetic
and boolean circuits, while the programmer remains
oblivious of all these cryptographic details. Indeed, \tool is the
first such cryptographic-cost aware compiler.

\subsubsection*{Scalability (secure code partitioning)} \mpc tools
often do not scale to large functionalities. The reason is that
most \mpc implementations use a circuit-like representation as an
intermediate language. Hence, the largest compute that can be done
securely is upper-bounded by the largest circuit that can fit in the
machine memory\footnote{Using swap and
disk space is feasible but it causes huge slowdown.}. This is a
show-stopper for applications like secure machine learning, secure
prediction, etc. that operate on large amounts of data.
%\aseem{Can we
%cite something here to support this claim?}
\tool addresses the scalability concern using a novel technique that
we call secure code partitioning (or partitioning in short). At
a high level, we decompose the original program into a sequence of small
sub-programs, which are then sequentially processed by \tool, while
appropriately threading the intermediate outputs
along. While this
addresses the scalability concern (the circuit
sizes of the sub-programs are now small enough to fit in the memory),
we still have to address
the security risk of revealing the intermediate outputs. \tool comes
to the rescue; it automatically inserts the required instrumentation
to ensure security of these intermediate outputs. As
we show in our
evaluation, partitioning allows us to program large applications
in \tool.

%% We provide details of partitioning and prove the
%% correctness and security of this transformation
%% in \sectionref{pipe}

\subsubsection*{Formal guarantees} We prove formal correctness and
security theorems for our compiler. The correctness theorem relates the
``trusted third party'' semantics of a source
program and the ``protocol'' semantics (the distributed \mpc
semantics that relies on circuit evaluation) of the corresponding
compiled program. The theorem
guarantees that for a well-typed source program, the two semantics
successfully terminate
(e.g. there are no array index out-of-bounds errors), with identical
observable outputs. For
the security theorem, we formally reduce the security
of our scheme, against semi-honest (or, ``honest but curious") adversaries,
 to the security of the  \mpc back-end. 
 We also prove a formal security theorem against semi-honest  adversaries for our partitioning
scheme (Section~\ref{sec:ld} and Section~\ref{sec:pipe}).

\subsubsection*{Evaluation} We have implemented \tool
using ABY~\cite{aby} as the
cryptographic back-end. We evaluate \tool by implementing a wide range
of secure prediction benchmarks including linear and na\"{i}ve bayes
classifiers, decision trees, deep neural networks, state-of-the-art
classifiers from Tensorflow~\cite{tensorflow}
and \bonsai~\cite{bonsai}, and also the matrix factorization example
from Nikolaenko et al.~\cite{valeriaMatrix}. Our results demonstrate
three key points. First, \tool makes it convenient for general
programmers to write \mpc protocols. E.g. we provide the first \mpc
implementation of \bonsai~\cite{bonsai}, and it was programmed in the
high-level \tool source language by a non-cryptographer. Second, the
performance
of the protocols
generated by \tool are comparable to or better than (up to 19x) their state-of-the-art specialized implementations. Finally, we
show that partitioning enables us to implement applications
that require more than 300 million gates (Section~\ref{sec:impl} and
Section~\ref{sec:eval}).

\subsubsection*{Related Work}
Before ABY~\cite{aby}, several works have proposed
combining secure computation protocols based on homomorphic
encryption and Yao's garbled circuits
(e.g. \cite{barni,blanton,brickell,franz,huang,valeriaMatrix,valeriaRidge,schropferK11}),
and some have also developed tools that allow writing such
combinations (e.g. \cite{bogdanov,lone,tasty,autoS}). However, as Demmler et
al.~\cite{aby} note, due
 to high conversion cost between
homomorphic encryption and Yao's garbled circuits, these combined
protocols do not gain much performance over a single
protocol. 
Additionally, these prior works provide informal languages
or libraries that lack formal semantics and static guarantees.
%Can be added to related work
% Kerschbaum et al.~\cite{autoS}
%provide a scheme to automatically assign homomorphic encryption or
%garbled circuits to each operators in a computation that has been expressed as a sequence of dyadic operations.
%They conjecture that the problem is NP hard and gave a linear programming based solution
%as well as a quadratic time greedy heuristic. \tool  uses a simpler constant time heuristic
%to assign arithmetic secret sharing or garbled circuits to sub-computations
%in a richer language with control flow statements such as loops and branches.
 We provide a detailed survey of related
work in \sectionref{related}.

%% The rest of the paper is structured as follows. We formalize \tool and
%% prove its correctness and security theorems in
%% Section~\ref{sec:ld}. Section~\ref{sec:pipe} details our partitioning
%% scheme and proves its security. Section~\ref{sec:impl}
%% presents the implementation details, and
%% Section~\ref{sec:eval} present a comprehensive evaluation of \tool. We
%% begin with an overview of \tool through an example in
%% Section~\ref{sec:ex}. %\nc{provided a pointer to sec 2 as well.}
%\aseem{Sure :), I did not add it because Sec. 2 was just next line, so
%  I thought it did not need a pointer.}



%% Most works cite garbled
%% circuit based two-party computation as the most practical approach so
%% far and use it as the baseline for comparison of performance. But, it
%% has been shown multiple times that the garbled circuits based approach
%% does not scale for realistic examples. Even for small benchmarks
%% (e.g., Na\"{i}ve Bayes) the size of garbled circuit can be
%% huge. Hence, comparisons are often done on artificial data sizes and
%% garbled circuits are shown to be 500x slower than hand-crafted
%% protocols \cite{shafindss}. This is cited as the motivation for need
%% for specialized protocols for different ML algorithms in secure
%% prediction such as linear regression \cite{shafindss}, decision
%% trees \cite{wu}, deep neural networks \cite{minionn} and matrix
%% factorization \cite{valeriaMatrix}. Our work is based on generic \mpc
%% and we show competitive performance with the specialized protocols
%% contrary to the popular belief. Hence, we believe \tool based
%% implementations can serve as the performance baseline for future works
%% on specialized protocols.


%% an intermdiate language (that is the output of our
%% compiler), and a circuit language for arithmetic and boolean
%% circuits.


%% We give a formal {\em type
%% system} for our language and prove useful properties. We prove that if
%% a program type checks in our language then it would run correctly,
%% e.g., there would not be any array index out of bounds error. The
%% programmer writes a single code describing the inputs from each party
%% and the function to be computed. Our compiler automatically generates
%% separate executables for the server and the client that can be
%% executed in the distributed setting. We prove the {\em correctness}
%% for this compilation, i.e., the outputs obtained in the distributed
%% execution are identical to the stand-alone execution by a trusted
%% third party. For details on formal
%% language, correctness and security theorems see \sectionref{ld}.






%% We demonstrate the generality of our framework by writing a wide range
%% of functionalities in \tool such as linear regression, deep neural
%% networks, decision trees and matrix factorization. 
%% Our evaluations show that automatically generated protocols
%% using \tool are competitive with tailor-made protocols
%% in \cite{shafindss,wu,barni,minionn,secureml,valeriaMatrix}
%% (see \sectionref{eval}). These features allow us to evaluate our
%% system state-of-the-art ML prediction algorithms,
%% e.g., \sectionref{eval} describes evaluation on ML models from a
%% recent ICML paper \cite{bonsai} and Tensorflow \cite{tensorflow}. \\


%% \tool is a programmer-friendly
%% framework to
%% express the function to be computed and uses a mix of arithmetic and
%% boolean \mpc back-ends. We illustrate the above example of $w^Tx
%% >b$ in our language in \figureref{ex-sml}. 
%% As is clear from the code, the programmer can remain completely
%% oblivious of the cryptographic back-end and underlying circuit
%% representation.
%% Also,  the programmer is not required to write keywords such as {\em
%% secret}
%% or {\em public} (that are required by languages like
%% Wysteria \cite{wysteria}).
%%  In contrast, \figureref{ex-aby} illustrates
%% the same example using the framework in ABY \cite{aby}. Here, as
%% already mentioned, the programmer needs to create shares and write the
%% low level gates
%% like $\mathtt{MUL, ADD, CONS, GT}$, etc \TODO{make these fonts
%% consistent}. The programmer is also required
%% to use the appropriate share conversion gates such as $\mathtt{A2Y}$ that
%% converts arithmetic shares to Yao-based boolean shares to be used in
%% comparison after the inner-product computation.
%% Moreover, it is easy to
%% maintain code in \tool and a small change to the functionality requires
%% only a small change in the code. In contrast, working with ABY, even
%% for a small change a developer might have to re-work the whole circuit
%% for efficiency reasons. We elaborate on these examples  in
%% \sectionref{ex}. \\

%% {\aseem{Not sure about this baseline paragraph. Did not touch it.}} 

%Finally, our framework \tool can serve as a meaningful baseline for future works on hand-crafted \mpc protocols for specific functionalities such as \textsc{MiniONN} \cite{minionn}. 
%Prior to our work, garbled circuit based two-party computation is cited as the most practical approach and used as the baseline for comparison of performance. But, it has been observed multiple times this approach does not even scale on realistic examples.

%% We have implemented \tool using ABY~\cite{aby} as the crypto backend.
%% We evaluate our system on wide range of benchmarks for secure
%% prediction \cite{shafindss,wu,barni,secureml,minionn,bonsai} using
%% linear classifier, decision trees and deep neural networks, matrix
%% factorization \cite{valeriaMatrix} and training of deep neural
%% network \cite{secureml}.



%% for secure
%% computation. We provide the first compiler where the developer writes
%% a high-level program and the compiler automatically generates an
%% efficient \mpc protocol that uses a
%% mix of arithmetic and boolean circuits. 
%% It is compatible with any implementation of \mpc that allows a mix of
%% arithmetic and boolean circuits.
%% We focus on semi-honest secure two-party computation and use
%% ABY \cite{aby} as the suitable cryptographic back-end.
%% We provide a formal type system and operational semantics for our
%% language and prove correctness and security of our compiler.
%% Our evaluations show that protocols generated
%% by our compiler match
%% the performance or outperform hand-crafted protocols in most cases (see
%% \sectionref{eval}). \TODO{Add some text to increase enthusiasm about
%% results.} Moreover, unlike previous systems such
%% as \cite{wysteria,aby,cbmcgc}, our system can scale to arbitrarily
%% large computations using our technique of {\em secure code
%% partitioning} (see below). \divya{I think we can skip Yao pipelining
%% here, right?}
%% Hence, we show that our framework achieves all four properties of
%% programmability, generality, performance, and scalability (see below
%% for details).

%% Prior to our work, there have been systems that provide only a subset of the above properties. For instance, using the framework of CBMC-GC \cite{cbmcgc} allows  functionalities to be expressed in high-level language C but it gives poor performance due to its usage of boolean circuits as underlying representation. Similarly, ABY \cite{aby} framework can potentially give good performance (due to mix of boolean and arithmetic circuits) but is not programmable. Yao's garbled circuits \cite{yao} work with boolean circuits and are the most widely used 2PC protocols but they  do not scale to large functions due to large memory requirement.
%% Huang et al. \cite{yao-pipe} gave the technique of pipelining Yao's garbled circuits that enabled scaling to large functions but still gives poor performance due to its use of boolean circuit representation. 
%% To summarize, none of the prior frameworks have been used to evaluate the benchmarks such as secure prediction using linear regression, decision trees, deep neural networks, etc. In fact, it was widely believed that generic 2PC would not scale to the secure prediction and hence, various protocols were hand-crafted for this task \cite{shafindss,wu,secureml,minionn}. Our framework contradicts this belief (see below).

%% In this work, we solve the design problem to provide a user friendly language that is expressive and can be compiled to efficient secure 2PC protocols.
%% Programmability in \tool and performance of generated protocols has enabled us to write large programs whose circuits  exhaust the memory of the system (~32 GB). To combat this issue, we give a technique called secure code partitioning (see below) that allows scaling of \tool to arbitrary size computations.

%% Below, we describe the salient features of \tool in more detail. \\




%% \aseem{Checkpointing here. Three more paragraphs to come, one for
%% summary of implementation and evaluation, one for related work, and
%% one for baseline argument.}

